\documentclass[12pt,a4paper,oneside,openright,titlepage]{book}
\usepackage[T1]{fontenc}
\usepackage[utf8]{inputenc}
\usepackage[italian]{babel}
\usepackage{graphicx}
\usepackage{listings}
\usepackage{mathtools}
\usepackage{enumerate}
\usepackage{forest}
\usepackage{booktabs}
\usepackage{caption}
\usepackage{multirow}
\usepackage{amsmath}
\pagestyle{headings}
\definecolor{javapurple}{rgb}{0.5,0,0.35} % keywords
\definecolor{javadocblue}{rgb}{0.25,0.35,0.75} % javadoc
\definecolor{javagreen}{rgb}{0.25,0.5,0.35} % comments
\definecolor{backcolour}{rgb}{0.95,0.95,0.92}%background
\lstset{language=Java,
	aboveskip=3mm,
	belowskip=3mm,
	columns=flexible,
	frame=tb,%single
	keywordstyle=\color{javapurple}\bfseries,
	stringstyle=\color{blue},
	commentstyle=\color{javagreen},
	morecomment=[s][\color{javadocblue}]{/**}{*/},
	numbers=left,%none
	numberstyle=\color{black},
	stepnumber=1,
	numbersep=3pt,
	tabsize=3,
	breaklines=true,
	showspaces=false,
	showstringspaces=false,
	backgroundcolor=\color{white},%\color{backcolour}
	basicstyle=\ttfamily\footnotesize}
\begin{document}
	\frontmatter
	\begin{titlepage}
\begin{center}
	{\LARGE Università degli Studi di Salerno}\par
	\vspace{0.5cm}
	{\Large Dipartimento di Informatica}\par
	\vspace{1cm}
	\includegraphics[height=160pt]{logounisa.png}\par
	\vspace{1cm}
	{\Large Corso di Laurea Magistrale in Informatica}\par
	\vspace{2cm}
	{\Huge GLL Parsing su linguaggi non lineari}\par
	\vspace{2cm}
\end{center}
\begin{flushleft}
	{\large\textbf{Relatore}}
	\hspace{8cm}
	{\large\textbf{Candidato}}\par
	\vspace{0.1cm}
	{\large Prof. Gennaro Costagliola}
	\hspace{4.5cm}
	{\large Mazzotta Fabio}%\\
	%{\normalsize Matr. 0522500518}\par
\end{flushleft}	
	\vspace{3.5cm}
\begin{center}
	{\large Anno Accademico 2018-2019}
\end{center}
\end{titlepage}
	\null\vspace{\stretch{1}}
\begin{flushright}
	\textit{Ai miei genitori.}\par
	\textit{Dedicato a chi ha creduto in me;}\\
	\textit{e a chi lotta ogni giorno e non si arrende.}\par
\end{flushright}
\vspace{\stretch{2}}\null

	\tableofcontents	
	\listoffigures
	\listoftables
	\mainmatter
	\chapter{Introduzione}
\section{Obiettivi}
Questa tesi di laurea descrive il funzionamento e l'implementazione del parsing \textbf{Generalizzato LL (GLL)} sui linguaggi non lineari. Il parsing GLL è un algoritmo di parsing top down che viene utilizzato per gestire tutte le grammatiche context-free che sono ambigue e ricorsive a sinistra. La caratteristica principale di questo algoritmo è che risulta essere un parser a \textbf{discesa ricorsiva} e ciò permette di avere il controllo del flusso sulle strutture della grammatica e risultano semplici da implementare e semplici da testare passo dopo passo attraverso il debugger. Questo parser è stato utilizzato per riconoscere linguaggi non lineari (bidimensionali) generati da grammatiche posizionali, ossia generalizzazioni di grammatiche context-free. La tesi è divisa in tre parti. Nella prima parte si cerca di illustrare come funziona il parsing LL, che rappresenta la base del parsing GLL, e i suoi limiti. Successivamente si  discuterà come estendere il parsing LL attraverso il parsing GLL, illustrandone i principi e le strutture dati che utilizza. Ciò viene descritto rispettivamente nel secondo e terzo capitolo. Nella seconda parte si analizzaranno le grammatiche posizionali. Questo argomento sarà trattato nel quarto capitolo. Nell'ultima parte si parlerà l'implementazione del parsing GLL applicato ad una grammatica posizionale. In particolare nel quinto capitolo si descriverà le varie  componenti software del parsing GLL, nel sesto capitolo si illustrerà come viene applicato il software del parsing GLL ad una grammatica posizionale e nel settimo capitolo si parlerà del tool utilizzato per testare il software del parsing GLL. Infine nell'ottavo capitolo si discuteranno i risultati ottenuti e gli sviluppi futuri.

	\chapter{Parsing LL(1)}
\section{Introduzione}
Il parsing, o analisi sintattica, è una fase di compilazione  che viene utilizzata per definire la sintassi di un linguaggio di programmazione. In altre parole definisce la forma di un programma corretto. Utilizza i token [1], ossia sequenze di caratteri dotate di significato restituite da un analizzatore lessicale (Lexer); per produrre una rappresentazione intermedia ad albero che rappresenta la struttura grammaticale dei token. Una tipica rappresentazione è l'\textit{albero sintattico}, o \textit{syntax tree} in cui un nodo interno rappresenta un'operazione mentre i figli rappresentano gli argomenti dell'operazione; infine, questo albero prodotto, viene passato alle restanti fasi del processo di compilazione. In figura 1 viene mostrato il funzionamento del parser.
\par
\vspace{0.5cm}	
\begin{tikzpicture}
	\tikzset{every node/.style={draw, rectangle, minimum size=25mm}}
	%\draw [-latex, bend right] (Lexer) ;	
	\node (Lexer) at (0.0,2.5) {Lexer};
	\node (Parser) at (3.0,2.5) {Parser};
	\node (Resto) at (7,2.5) {Resto del Compilatore};
	\end{tikzpicture}
\section{Parsing top down}
%\subsection{paragrafo}
jrjfvrlf
	\chapter{GLL Parsing}
\section{Introduzione}
Nel capitolo precedente abbiamo discusso i concetti e il funzionamento del parsing su grammatiche LL(1). In questo capitolo discuteremo di un estenzione di questo parsing, chiamato \textbf{Parsing LL Generalizzato (GLL)} e vedremo come questo parser supera i limiti che hanno i parser LL(1). Mostreremo i concetti base di questo parsing e del suo funzionamento
\section{Stack e descrittori elementari}
In questo paragrafo discuteremo del funzionamento base del GLL Parsing. Data la seguente grammatica:
\begin{align}\label{gram3}
	S & \to ASd \mid BS \mid \epsilon \notag \\
	A & \to a \mid c \notag \\
	B & \to a \mid b 
\end{align}
Un parser a discesa ricorsiva \cite{pubblicazione: scott} è composto dalle seguenti funzioni: $p_S$(), $p_A$(), $p_B$(), la funzione principale \textit{main}() e la funzione per segnalare gli errori \textit{error}(). Ogni funzione contiene codice per ogni alternativa, $\alpha$, e verificano il simbolo corrente della stringa in input appartiene a FIRST($\alpha$) o al FOLLOW($\alpha$). La stringa in input viene rappresentata come un array globale \textit{I} di lunghezza \textit{m}+1, dove \textit{I}[\textit{m}]=$\$$, segnala la fine della stringa. L'implementazione del parser viene rappresentata di seguito.\\
main()$\{$ \textit{i} = 0 \par
\hspace{1cm}\textbf{if}(\textit{I}[\textit{i}] $\in$ $\{$\textit{a,b,c,d,$\$$}$\}$)$\{$ $p_S$(); \textbf{else} \textit{error}();\par
\hspace{1cm}\textbf{if}(\textit{I}[\textit{i}] = $\$$)$\{$ report success $\}$ \textbf{else} \textit{error}()\\	
$\}$\\
$p_S$()$\{$ \\
\textbf{if}(\textit{I}[\textit{i}] $\in$ $\{$\textit{a,c}$\}$)$\{$ $p_A$(); $p_S$(); \textbf{if}(\textit{I}[\textit{i}]=\textit{d})$\{$ \textit{i}=\textit{i}+1$\}$ \textbf{else} \textit{error}();$\}$\par
\hspace{0.5cm}\textbf{if}(\textit{I}[\textit{i}] $\in$ $\{$a,b$\}$)$\{$ $p_B$(); $p_S$(); $\}$ $\}$\\
$p_A$()$\{$ \par
\hspace{0.5cm}\textbf{if}(\textit{I}[\textit{i}]=\textit{a})$\{$ \textit{i}=\textit{i}+1;$\}$\par \hspace{0.5cm}\textbf{else} \textbf{if}(\textit{I}[\textit{i}] = \textit{c})$\{$ \textit{i}=\textit{i}+1 $\}$ \textbf{else} \textit{error}(); $\}$\\
$p_B$()$\{$ \par \hspace{0.5cm}\textbf{if}(\textit{I}[\textit{i}]=\textit{a})$\{$ \textit{i}=\textit{i}+1;$\}$\par \hspace{0.5cm}\textbf{else} \textbf{if}(\textit{I}[\textit{i}] = \textit{b})$\{$ \textit{i}=\textit{i}+1 $\}$ \textbf{else} \textit{error}(); $\}$\par
\vspace{0.3cm}
Questa è la tabella di parsing della grammatica \ref{gram3}.
\begin{table}[hbpb]
	\centering
	\label{tabellaparsingNLL1}
	\begin{tabular}{cccccc} 
		\toprule
		%\multirow{2}*{\textbf{Non Terminale}} & %\multicolumn{4}{c}{\textbf{Simbolo d'ingresso}} \\ 
		%\cmidrule(lr){2-4}
		& a & b & c & d & $\$$ \\ 
		\midrule
		\textit{S} 	& \textit{S}$\to$\textit{ASd} $\mid$ \textit{BS} & \textit{S}$\to$\textit{BS}&\textit{S}$\to$ \textit{ASd}&\textit{S}$\to$$\epsilon$& \textit{S}$\to$$\epsilon$\\ 
		\textit{A} & \textit{A}$\to$\textit{a}&  & \textit{A}$\to$\textit{c}\\ 
		\textit{B} & \textit{B}$\to$\textit{a} & \textit{B}$\to$\textit{b}&   \\ 
		\bottomrule
	\end{tabular}
	\caption{\textit{Tabella di parsing della grammatica }\ref{gram3}}
\end{table} \par
Da quello che si può notare dalla tabella \ref{gram3} questa grammatica non è LL(1) in quanto è presente un conflitto e di conseguenza l'algoritmo implementato non funziona correttamente. Affinchè l'algoritmo funzioni correttamente è necessario aggiungere il non-determinismo. Per fare ciò dobbiamo convertire le chiamate a funzioni con operazioni di \textbf{push} su  uno stack e utilizzare i \textbf{goto}. Poi partizioniamo in varie parti i corpi delle funzioni il cui non-terminale non è LL(1) ed attribuiamo un etichetta ad ogni partizione. In questo caso abbiamo più partizioni per \textit{S}. Per registrare le possibili scelte che il parser può fare per sostituire un non-terminale utilizziamo dei \textbf{descrittori} all'interno dell'algoritmo a discesa ricorsiva e sostituiamo il punto in cui termina l'algoritmo con l'esecuzione di un descrittore successivo. Le funzioni d'errore vengono sostituite con l'esecuzione di descrittori successivi. Il nuovo punto di termine sarà quando non esistono più descrittori da eseguire. Formalmente un \textbf{descrittore elementare} è una tripla (\textbf{L,s,j}) dove \textbf{L} è un etichetta, \textbf{s} è uno stack e \textbf{j} è la posizione nell'array \textit{I}. Quessti descrittori li manteniamo in un insieme \textbf{R}. Ogni volta che si verifica la fine di una funzione di parsing e ad ogni punto in cui è presente un terminale non LL(1) (quindi siamo in presenza di non-determinismo) all'interno dell'algoritmo, creiamo un nuovo descrittore che è formato dall'etichetta in cima allo stack corrente. Quando l'algoritmo di parsing trova un simbolo del'input \textit{I}[\textit{i}] diciamo che l'etichetta \textit{L} in cima allo stack è estratto dallo stack \textit{s}=[\textit{s$^{'}$},\textit{L}] e (\textit{L,s$^{'}$,i}) viene aggiunta a \textbf{R}. Questa azione viene denotata con la funzione \textit{pop(s,i,\textbf{R})}. Dopo aver fatto ciò rimuoviamo il descrittore (\textit{L$^{'}$,t,j}) da \textbf{R} e l'algoritmo riparte dall'etichetta \textit{L$^{'}$}, con stack \textit{t} e con il simbolo in input \textit{I}[\textit{j}]. L'algoritmo termina quando l'insieme \textbf{R} è vuoto. Useremo la notazione \textit{L}$^{k}$ per unire l'etichetta \textit{L} e l'indice \textit{k} che indica il simbolo corrente nell'input \textit{I}; mentre lo stack vuoto viene denotato con []. Lo stack \textit{s} viene aggiornato con la funzione \textit{push(s,L$^{k}$)}; questa funzione non fa altro che aggiungere  l'elemento \textit{L}$^{k}$ in cima allo stack. Di seguito viene presentato l'algoritmo.\par
\vspace{0.5cm}
\hspace{0.4cm}i=0; \textbf{R}=$\emptyset$; s=[L$_0^{0}$];\\
$L_S$: \textbf{if}(\textit{I}[\textit{i}] $\in$ $\{$\textit{a,c}$\}$) add($L_{S1}$,s,i) to \textbf{R} \par
\hspace{0.4cm}\textbf{if}(\textit{I}[\textit{i}] $\in$ $\{$\textit{a,b}$\}$) add($L_{S2}$,s,i) to \textbf{R}\par
\hspace{0.4cm}\textbf{if}(\textit{I}[\textit{i}] $\in$ $\{$\textit{d,$\$$}$\}$) add($L_{S3}$,s,i) to \textbf{R} \\
$L_0$: \textbf{if}(\textbf{R}$\ne$$\emptyset$)$\{$ remove(L, $s_1$, j) from \textbf{R} \par
\hspace{0.4cm}\textbf{if}(L = $L_0$ and $s_1$=[ ] and j=|I|) report success \par
\hspace{0.4cm}\textbf{else}$\{$ s=$s_1$; i=j; \textbf{goto} L $\}$ \\
%\vspace{0.5cm}
$L_{S1}$:\textit{push}(s,L$_1^{i}$); \textbf{goto} $L_A$\\
$L_1$:  \textit{push}(s,L$_2^{i}$); \textbf{goto} $L_S$\\
$L_2$:  \textbf{if}(\textit{I}[\textit{i}] = a)$\{$ i=i+1; \textit{pop}(s, i, \textbf{R});$\}$ \textbf{goto} $L_0$ \\
$L_{S2}$:\textit{push}(s,L$_3^{i}$); \textbf{goto} $L_B$\\
$L_3$:  \textit{push}(s,L$_4^{i}$); \textbf{goto} $L_S$\\
$L_4$: \textit{pop}(s, i, \textbf{R}); \textbf{goto} $L_0$ \\
$L_{S3}$:\textit{pop}(s, i, \textbf{R}); \textbf{goto} $L_0$ \\
$L_A$:  \textbf{if}(\textit{I}[\textit{i}] = a)$\{$ i=i+1; \textit{pop}(s, i, \textbf{R});$\}$ \textbf{goto} $L_0$ $\}$ \par
\hspace{0.2cm} \textbf{else}$\{$ \textbf{if}(\textit{I}[\textit{i}] = c)$\{$ i=i+1; \textit{pop}(s, i, \textbf{R});$\}$ \par 
\hspace{0.3cm}\textbf{goto} $L_0$ $\}$ \\
$L_B$:  \textbf{if}(\textit{I}[\textit{i}] = a)$\{$ i=i+1; \textit{pop}(s, i, \textbf{R});$\}$ \textbf{goto} $L_0$ $\}$ \par
\hspace{0.2cm} \textbf{else}$\{$ \textbf{if}(\textit{I}[\textit{i}] = b)$\{$ i=i+1; \textit{pop}(s, i, \textbf{R});$\}$ \par
\hspace{0.3cm}\textbf{goto} $L_0$ $\}$ \\
\section{Costruzione del GSS}
joijo
\subsection{Graph structured stacks}
yuyuy
\subsection{Insiemi U e P}
yuyuyu
\section{Definizione GLL Parsing}
nklnlknkn
\subsection{Funzioni Fondamentali}
kiuiu
\subsection{Gestione degli item}
opopokp
\subsection{Gestione dei simboli non-terminali}
lòlòkbj
\subsection{Shared packed parse forests}
yuyuy
\section{Costruzione del GLL Parser}
eeeededed
	\chapter{GLL Parsing Posizionale}
\section{Introduzione}
In questo capitolo introduciamo un estensione del GLL parsing, il \textbf{GLL Parsing Posizionale}. Il GLL Parsing Posizionale permette di trattare le \textit{grammatiche posizionali} che producono i cosidetti \textbf{linguaggi non lineari}. Queste grammatiche presentano produzioni contenenti \textit{relazioni spaziali}, ossia relazioni che indicano come deve essere letto l'input. Illustreremo la definizione base di questa nuova grammatica e il funzionamento base delle relazioni spaziali.
\section{Definizione formale}
Una \textbf{grammatica posizionale context-free} è sestupla \cite{pubblicazione: tomita} i cui elementi sono:
\begin{enumerate}
	\item \textbf{Non-Terminali (N)}: variabili sintattiche che denotano un insieme di stringhe.
	\item \textbf{Terminali (T)}: simboli di base che definiscono il linguaggio.
	\item \textbf{Simbolo iniziale (S)}: è un non-terminale e l'insieme di stringhe che esso denota coincide con l'intero linguaggio generato dalla grammatica.
	\item \textbf{Produzioni (P)}: regole che definiscono come possono essere combinati i terminali e i non-terminali.
	\item \textbf{Relazioni Spaziali (POS)}: relazioni che danno informazioni sulle posizioni di spostamento da effettuare per la lettura dell'input.
	\item \textbf{Regole di valutazione (PE)}: regole usate per determinare come deve essere effettuato lo spostamento sull'input.
\end{enumerate}
Ogni produzione presenta la seguente forma:
\begin{align}
	A \to \alpha_{1} REL_1 \alpha_{2} REL_2 \dots REL_{m-1} \alpha_{n} \quad m \ge 1 \notag 
\end{align}
dove \textit{A} $\in$ N, ogni $\alpha_i$ $\in$ P ed ogni \textit{REL}$_1$ $\in$ POS.\par
\vspace{0.3cm}
\noindent Ogni relazione spaziale \textit{REL}$_1$ dà informazioni sulla posizione relativa ad $\alpha_{i+1}$ rispetto ad $\alpha_{i+1}$. Nelle grammatiche tradizionali abbiamo una sola relazione spaziale che è data dalla concatenazione della stringa, nelle grammatiche posizionali possiamo definire altre relazioni spaziali ed in questo modo possiamo usarli per descrivere i linguaggi bidimensionali. In questi linguaggi, le informazioni posizionali vengono usate dal lexer per leggere il prossimo token.
\subsection{Relazioni spaziali}
L'insieme di relazioni spaziali \cite{pubblicazione: tomita} possono essere suddivisi in 3 sottoinsiemi: \textit{type}-2, \textit{type}-1, \textit{type}-0. Ogni sottoinsieme include quattro relazioni spaziali direzionali che sono: sopra, sotto, destra e sinistra. Ogni direzione può essere descritta da una funzione che prende in input una coordinata nel piano cartesiano e restituisce in output un insieme di posizioni. Infatti, se diamo in input una posizione \textit{p} nel piano, la relazione spaziale $\to$ in type-2 applicata a \textit{p} restituisce le prime posizioni a destra di \textit{p}; la relazione spaziale $\to$ in type-1 restituisce un insieme di posizioni a destra di \textit{p} e nella stessa linea di \textit{p}; la relazione spaziale $\to$ in type-0 restituisce un insieme di posizioni a destra di \textit{p} anche se non sono nella stessa posizione di \textit{p}. Qui di seguito mostriamo le definizioni dei tre tipi di relazioni spaziali ed illustreremo solo le funzioni descritte; le altre possono essere ricavate in maniera molto simile
\begin{flushleft}
	\textbf{Type-2}  \\
	$\to$   spostamento  di un movimento a destra: (\textit{x}, \textit{y}) $\to$ (\textit{x}+1, \textit{y})\\
	$\gets$   spostamento  di un movimento a sinistra \\
	$\uparrow$   spostamento  di un movimento in alto \\
	$\downarrow$   spostamento  di un movimento in basso \par
	\vspace{0.2cm}\noindent \textbf{Type-1}  \\
	$\to_1$   spostamento a destra sulla stessa linea:\par 
	\hspace{0.5cm}(\textit{x}, \textit{y}) $\to$ $\{$$\to^{(n)}$  (\textit{x},\textit{y}) $\mid$ n=1,2,$\dots$$\}$ \\
	$\gets_1$ spostamento a sinistra sulla stessa linea\\
	$\uparrow_1$ spostamento in alto sulla stessa linea\\
	$\downarrow_1$ spostamento in basso sulla stessa linea \par
	\vspace{0.2cm}\noindent \textbf{Type-0}  \\
	$\to_0$  spostamento a destra: \par 
	(\textit{x},\textit{y}) $\to$   $\to_0$(\textit{x},\textit{y})  $\cup$ $\downarrow_1$(\textit{x}$^{'}$,\textit{y}$^{'}$) $\cup$ $\to_1$ (\textit{x}$^{'}$, \textit{y}$^{'}$) dove (\textit{x}$^{'}$, \textit{y}$^{'}$) cambia in$\to_0$(\textit{x},\textit{y})\\
	$\gets_0$ spostamento  a sinistra \\
	$\uparrow_0$ spostamento in alto \\
	$\downarrow_0$ spostamento in basso 
\end{flushleft}
Nei nostri esempi utilizzeremo le relazioni spaziali HOR e VER che corrispondono alle relazioni spaziali $\to_0$ e $\downarrow_0$ - $\to_0$, rispettivamente, dove $\downarrow_0$ - $\to_0$ è un insieme che risulta essere la differenza tra l'insieme di posizioni generate da $\downarrow_0$ e $\to_0$ quando sono applicate alla stessa posizione. 
\subsection{Regole di valutazioni}
Una regola di valutazione PE \cite{pubblicazione: tomita} è una funzione che prende in input una stringa del tipo 
\begin{align}
	\textit{p}_1 REL_1 \textit{p}_2 REL_2 \dots REL_{m-1} \quad    m \ge 1 \notag
\end{align}
dove ogni \textit{p}$_i$ è una posizione ogni \textit{REL}$_i$ è una relazione spaziale, il suo output è una \textbf{immagine} dove gli elementi \textit{p}$_1$, \textit{p}$_2$,$\dots$, \textit{p}$_n$, sono disposti nello spazio in questo modo:
\begin{align}
	\textit{p}_i+1 \in REL_i (\textit{p}_i)  \quad    1\le \textit{i} \le \textit{m}-1 \notag
\end{align}
Le regole di valutazione delle relazioni spaziali sono state pensate per essere una sequenza da sinistra a destra. Diciamo che una regola di valutazione è \textbf{semplice} se non presenta effetti collaterali. \\
Un esempio di applicazione di semplici regole di valutazione sono i seguenti:
\begin{align}
	PE(a \to b \to c \to d) \quad \quad = a b c d \notag \\ 
		PE(a \quad VER \quad b \quad HOR \quad c) \quad  = \quad a \notag \\ 
		                                                              b \quad  c \notag 
\end{align}
\textbf{Derivazioni}\\
Denotiano $\alpha$ $\overset{*}{\Rightarrow}$ $\beta$ (si legge  $\alpha$ deriva inb zero o più passi $\beta$) se esiste una stringa $\alpha_0$$\alpha_1$$\dots$$\alpha_m$ (m$\ge$0) tale che
\begin{align}
	\alpha \Rightarrow \alpha_0 \Rightarrow \alpha_1 \Rightarrow \dots \Rightarrow  \alpha_m \quad = \quad \beta  \notag
\end{align}
La sequenza $\alpha_0$$\alpha_1$$\dots$$\alpha_m$ è chiamata \textbf{derivazione} di $\beta$ da $\alpha$. Una \textbf{forma sentenziale posizionale} è una stringa $\alpha$ tale che S $\overset{*}{\Rightarrow}$ $\alpha$. Una \textbf{frase posizionale} è una forma sentenziale posizionale con soli simboli terminali. Una \textbf{forma pittorica} è la valutazione di una forma sentenziale posizionale. Un \textbf{immagine} è una forma pittorica che non contiene non-terminali. Il \textbf{linguaggio pittorico L(G)} definito da una grammatica posizionale G è l'insieme delle sue immagini. Un esempio di grammatica posizionale è mostrato di seguito.
\begin{center}
	\textbf{N} = \{E,T,F\} \\
	\textbf{S} = E  \\
	\textbf{T} =  \{\textit{+}, \textit{hbar}, \textit{(}, \textit{)}, \textit{id}\} \\
	\textbf{POS}  = \{HOR, VER\}\\
	\textbf{PE} non è una semplice regola di valutazione\\
	\textbf{P}   = \{ E $\to$ E \textit{HOR} \textit{+}\textit{HOR} T $\mid$ T \par 
	         \hspace{1.1cm}T $\to$ \textit{VER} \textit{hbar} \textit{VER} F $\mid$ F \par 
	         \hspace{1.1cm} F $\to$ (\textit{HOR} E \textit{HOR}) $\mid$ \textit{id}  \}   
\end{center}
Una frase posizionale di questa grammatica è:
\begin{center}
	\textit{id} \textit{HOR} \textit{+} \textit{HOR} ( \textit{HOR} \textit{id} \textit{HOR} + \textit{HOR} \textit{id} \textit{HOR}) \textit{VER} \textit{hbar} \textit{VER} \textit{id} \textit{HOR} + \textit{HOR} + \textit{id}
\end{center}
Da questa immagine possiamo ottenere una delle possibili immagini:
\begin{center}
	 \textit{id} + $\frac{(\textit{id} + \textit{id})}{\textit{id}}$ +\textit{id}
\end{center}
\section{Parser Posizionali}
In questo paragrafo descriveremo il funzionamento dei parser posizionali e il modo che usano per gestire l'input.
\subsection{Gestione dell'input}
L'input che viene dato in pasto al parser è un immagine simbolica e ogni simbolo contenuto in esso è un token. L'input viene rappresentato da un array \textit{X} dove vengono memorizzati i token e da una matrice M che rappresenta l'input in base alla loro posizione spaziale e sono memorizzati con le posizioni \textit{i-esime} dell'array \textit{X}. Per segnalare la fine della stringa si aggiunge il simbolo $\$$ sia alla fine dell'array \textit{X} e sia nella matrice M.
\subsection{Gestione degli operatori spaziali}
Per ogni relazione spaziale definiamo un operatore spaziale con lo stesso nome. Ogni qualvolta il parser lo trova viene chiamata la funzione \textit{getNextToken()} che prende in input l'indice dell'array \textit{X} dell'ultimo token visto, l'operatore spaziale e un array \textit{Y} che contiene i token già visti. Restituisce il token successivo, dopo aver consultato la matrice \textit{M}. La ricerca sulla matrice avviene in questo modo:
\begin{itemize}
	\item Se l'operatore spaziale è \textit{HOR} (>) il calcolo del token successivo avviene in questo modo: si cerca un token non visto nelle colonne a destra della matrice M dell'ultimo token visto.
	\item Se invece l'operatore spaziale è \textit{VER} (<) il calcolo del token successivo avviene cercando un token non ancora visto nella righe successive dell'ultimo token visto all'interno della matrice \textit{M}.
\end{itemize}
La ricerca avviene usando l'array dei token visti e si restituisce il primo token che non è stato ancora visto.
	\chapter{Implementazione del GLL parsing}
\section{Introduzione}
In questo capitolo illustreremo come è stato implementato il GLL Parsing. Parleremo le componenti software che lo compongono ed illustremo come il software creato è stato utilizzato per un linguaggio lineare.
\section{Le componenti del sistema}
Mettere il diagramma delle classi del parser
\section{Gli elementi dell'insiemi R, P e U}
mettere classi insiemi U P R
\section{La classe GLL Parser}
mettere la classe con il main che indica il gll parsing

	\chapter{GLL parsing posizionale}
\section{Introduzione}
Nel capitolo seguente vedremo come è stato implementato il GLL parsing su una grammatica posizionale. Descriveremo come sono gestiti gli operatori spaziali e l'input. 
\section{Gestione dell'input}
La gestione dell'input è stato definito nella classe \textbf{\textit{InputDataset.java}}.
\lstinputlisting{C:/Users/fabio/Documents/GitHub/gll-parsing/src/dataset/InputDataset.java}
\section{La classe GLLParserBidimensionale}
mettere la classe del gll bidimensionale
	\chapter{ParVis}
\section{Introduzione}
In questo capitolo illustremo ParVis, un tool grafico che viene utilizzato per rappresentare graficamente le operazioni che vengono effettuate dai parser. Questo tool offre la possibilità di illustrare graficamente le strutture dati usate dal parser, l'albero sintattico, il testo in input su cui si sta facendo il parsing, il simbolo e lo stato corrente che si sta analizzando. Nei paragrafi successivi illustreremo le componenti basi utilizzate da ParVis, il suo funzionamento e le componenti create per rappresentare graficamente il GLL Parsing.
\section{Il prompt dei comandi}
Il \textbf{prompt dei comandi} è la componente fondamentale che viene usata per gestire le impostazioni grafiche di ParVis, l'esecuzione e il caricamento delle operazioni da illustrare. Per avviare l'esecuzione di un illustrazione grafica è necessario caricare il file \textbf{\textit{log.json}}. Questo file viene generato dopo l'esecuzione del parser e contiene le operazioni di log effettuate. Nella figura 7.1 viene illustrato il prompt dei comandi. Presenta un menu "File" e "Windows" che sono utilizzati, rispettivamente, per caricare il file \textit{log.json} e per settare le impostazioni grafiche delle finestre. Poi vi sono tre pulsanti dove i primi due, a partire da sinistra, sono utilizzati per eseguire un operazione in avanti e all'indietro; mentre il terzo viene usato per automatizzare l'esecuzione delle operazioni. Nella parte principale del prompt dei comandi è presente una lista che contiene i vari item delle produzioni, chiamati \textbf{stati}. Questi stati sono raccolti in menu a tendina poichè contengono le varie operazioni effettuate per gestire un item. Possono essere aperti e chiusi cliccando o su singolo stato o con i pulsanti in alto alla lista. Uno stato evidenziato indica che ParVis sta visualizzando le operazioni di quello stato, se lo si apre vedremo evidenziate le sue operazioni all'interno. In basso alla lista è presente uno slider che viene utilizzato per eseguire più velocemente le operazioni di visualizzazione.\par
\begin{figure}[hbpb]\label{prompt}
	{\includegraphics[height=350pt,width=420pt,scale=0.1]{files/ParVisCommand.png}}
	\caption{\textit{Prompt dei comandi}}
\end{figure}
\section{Le componenti grafiche del GLL Parsing}
In questo paragrafo verranno illustrate le componenti grafiche realizzate per il GLL Parsing. Queste componenti sono state implementate nel file \textbf{\textit{jsonreader.js}} e ne sono state definite le impostazioni grafiche. Ogni finestra presenta un toolitip in cui vengono descritte le informazioni contenute da ogni elemento grafico. Ogni finestra subisce continue modifiche per ogni operazione che si sta eseguendo.\par
\begin{figure}[hbpb]\label{insiemeU}
	\centering
	{\includegraphics[height=200pt,width=320pt,scale=0.1]{files/InsiemeU.png}}
	\caption{\textit{Insieme U}}
\end{figure}
\noindent L'insieme U (fig. 7.2) viene rappresentato come una lista di nodi collegati uno dietro l'altro. Riporta le seguenti informazioni: il nome dello stato e il nome del nodo del GSS.\par
\begin{figure}[h]\label{input}
	\centering
	{\includegraphics[height=180pt,width=120pt,scale=0.1]{files/Input.png}}
	\caption{\textit{Testo in Input}}
\end{figure}
\noindent L'input (fig. 7.3) viene rappresentato allo stesso modo in cui è stato definito nel file d'input. Sull'input vengono utilizzati tre colori: il nero indica un simbolo non ancora letto, il rosso indica un simbolo che il parser vuole trovare ed il blu indica un simbolo che è stato trovato dal parser.\par
\begin{figure}[hbpb]\label{insiemeP}
	\centering
	{\includegraphics[height=190pt,width=320pt,scale=0.1]{files/InsiemeP.png}}
	\caption{\textit{Insieme P}}
\end{figure}
\noindent L'insieme P (fig. 7.4) è rappresentato come una lista di nodi. Sono state rappresentate le seguenti informazioni: il nodo del GSS, la posizione del simbolo da trovare e il nodo dell'SPPF.\par
\begin{figure}[hbpb]\label{gss}
	\centering
	{\includegraphics[height=190pt,width=250pt,scale=0.1]{files/GSS.png}}
	\caption{\textit{GSS}}
\end{figure}
\noindent La figura 7.5 mostra la rappresentazione grafica del GSS. Viene rappresentato come un grafo diretto aciclico (DAG) che non contiene cicli. Ogni nodo contiene le informazioni sul simbolo che si vuole trovare e lo stato che identifica l'item da processare. Il loro nome viene rappresentato in base alle notazioni descritte al capitolo 5.\par
\begin{figure}[hbpb]\label{SPPF}
	{\includegraphics[height=260pt,width=420pt,scale=0.1]{files/SPPF.png}}
	\caption{\textit{SPPF}}
\end{figure}
\noindent L'SPPF (fig. 7.6) viene rappresentato con vari tipi di nodi: i nodi arancioni vengono usati per identificare i non-terminali, i nodi verdi vengono usati per i terminali, i nodi grigi rappresentano tutti quei nodi che appartengono a produzioni che non sono state completate per intero, nel tooltip vengono segnati come \textit{Not Valid}. Le informazioni riportate sono: il tipo del nodo, la posizione che occupa nell'input (solo per i terminali), il nome del simbolo, l'item di produzione a cui appartiene.\par 
\begin{figure}[hbpb]\label{currentState}
	\centering
	{\includegraphics[height=110pt,width=110pt,scale=0.1]{files/CurrentState.png}}
	\caption{\textit{Stato Corrente}}
\end{figure}
\noindent La finestra in figura 7.7 viene utilizzata per rappresentare lo stato corrente su cui il parser sta eseguendo le operazioni. \'E composta da un nodo che indica il nome dello stato e l'item che sta processando.\par
\begin{figure}[hbpb]\label{DumpFile}
	\centering
	{\includegraphics[height=250pt,width=330pt,scale=0.1]{files/DumpFile.png}}
	\caption{\textit{Informazioni sulla grammatica e sugli stati del parser}}
\end{figure}
\noindent La finestra in figura 7.8 viene utilizzata per rappresentare le informazioni inerenti alla grammatica che implementa il parser. Vengono descritti i terminali, i non-terminali, le produzioni, il simbolo iniziale (che corrisponde sempre alla prima produzione) e i vari item (stati) che vengono gestiti dal parser. Ogni volta che il parser esegue le operazioni di uno stato viene colorato di rosso.\par 
\begin{figure}[hbpb]\label{InsiemeR}
	\centering
	{\includegraphics[height=130pt,width=250pt,scale=0.1]{files/InsiemeR.png}}
	\caption{\textit{Insieme R}}
\end{figure}
\noindent L'insieme R (fig. 7.9) viene rappresentato come una lista di nodi. Ogni nodo riporta le seguente informazioni: il nome dello stato, la posizione del simbolo da trovare, il nodo del GSS e il nodo dell'SPPF.
	\chapter{Conclusioni}
\section{Obiettivi raggiunti}
Gli obiettivi di questa prima parte di lavoro era quello di introdurre il parsing top down descrivendo il parsing LL(1) e ne è stato delineato il suo funzionamento e i suoi limiti, in quanto non riesce a gestire grammatiche ambigue e ricorsive. Per superare questi limiti abbiamo introdotto il parsing GLL che utilizza il non determinismo per gestire grammatiche ambigue e ricorsive. Il suo utilizzo permette di gestire tutti i conflitti dell'LL(1) processando tutte sostituzioni possibili per un non terminale. Questo parsing usa una struttura dati molto potente per gestire la computazione del non determinismo ed è il GSS. Il risultato prodotto da questo parsing è l'Sppf, un albero che combina tutti i risultati dei vari parser creati dalla computazione del GLL. L'obiettivo finale di questo lavoro ha portato a creare un estensione del GLL Parsing per grammatiche posizionali utilizzando sempre lo stesso algoritmo per processare i non-terminali e gli item delle varie produzioni, però ne è stata modificata la gestione della lettura dei token successivi in quanto non avveniva più in maniera lineare, dove i token vengono letti in successione, ma avviene in base alle regole definite dagli operatori spaziali che possono diversi per ogni grammatica posizionale.
	\backmatter
	\addcontentsline{toc}{chapter}{\\Bibliografia}
	\begin{thebibliography}{9}
\bibitem{eco:tesi}
Eco, Umberto (1977),
\emph{Come si fa una tesi di
laurea}, Bompiani, Milano.
\bibitem{mori:tesi}
Mori, Lapo Filippo (2007),
“Scrivere la tesi di laurea
con latex
\end{thebibliography}
\end{document}