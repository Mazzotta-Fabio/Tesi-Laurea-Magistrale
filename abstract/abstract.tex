\documentclass[12pt]{article}
\usepackage[T1]{fontenc}
\usepackage[utf8]{inputenc}
\usepackage[italian]{babel}
\begin{document}
	\begin{center}
		{\huge\bf Abstract}\par
	\end{center}
	La tesi descrive il funzionamento e l'implementazione del GLL Parsing applicato ai linguaggi non lineari. Il Generalised LL (GLL) parsing è un parser generalizzato top-down che viene utilizzato per gestire tutte le grammatiche context-free comprese quelle che risultano essere ambigue e ricorsive. Questo parser risulta essere molto più potente dei tradizionali parser LL(1) in quanto riesce a superare i conflitti di sostituzione per un non-terminale presenti nella tabella di parsing per un simbolo in ingresso. Infatti, durante la costruzione dell'albero sintattico, il GLL parsing sostituisce un non-terminale con tutte le produzioni in conflitto. Per poter fare ciò il parser usa il principio del non determinismo, cioè vengono creati più flussi di computazione per ogni conflitto di sostituzione. Per combinare i vari stack usati dai vari flussi di computazione si usa il graph structured stack (GSS). Il risultato ottenuto dal GLL parsing è lo shared packed parse forest (SPPF), una struttura dati in cui i nodi e gli archi dei vari alberi sintattici prodotti dai vari flussi di computazione vengono raccolti e condivisi in un unica struttura dati. Una caratteristica di questo parser è che risulta essere un parser a discesa ricorsiva, ciò permette un maggiore controllo sulla struttura della grammatica e di conseguenza ne facilita l'implementazione e il testing poichè risulta possibile testare ogni singola istruzione attraverso l'utilizzo del debugger. L'obiettivo raggiunto è stato quello di estendere la computazione del GLL parsing a grammatiche posizionali che producono i cosidetti linguaggi non lineari. Queste grammatiche rappresentano un estensione delle grammatiche context-free dove in aggiunta hanno delle relazioni spaziali. Queste relazioni sono utilizzate per stabilire come devono essere letti i simboli successivi dal testo in input; pertanto i simboli non vengono letti solo da sinistra verso destra come nelle grammatiche context-free, ma vengono letti in più direzioni e la direzione esatta viene stabilita dalla relazione spaziale. Per gestire queste grammatiche, il parser GLL userà sempre gli stessi principi di funzionamento usati per le grammatiche context-free, però leggerà i simboli successivi in base alle direzioni stabilite dalle relazioni spaziali incontrate durante la computazione.
\end{document}

