\chapter{Implementazione del GLL Posizionale}
\section{Introduzione}
Nel capitolo seguente vedremo come è stato implementato il GLL parsing sulle grammatiche posizionali. Descriveremo come sono state gestite le relazioni spaziali e come è stato applicato su due grammatiche posizionali. Per il resto la gestione dei non terminali e degli item delle varie produzioni sono state gestite allo stesso modo che vale per il GLL parsing lineare.
\section{GLL Parsing su espressioni aritmetiche}
In questo paragrafo discutiamo di come è stato implementato e gestito il GLL Parsing sulla grammmatica posizionale delle espressioni aritmetiche. La grammatica è la seguente:
\begin{align}\label{gramPos1}
E & \to T > + > E \mid  T \notag \\
T & \to F < hbar < T \mid F \notag \\
F & \to ( > E > ) \mid id 
\end{align}
Il simbolo \textbf{>} e \textbf{<} sono relazioni spaziali che indicano rispettivamente di spostarsi in orizzontale e in verticale per leggere il simbolo successivo.
\subsection{Gestione dell'input}
La gestione dell'input è stato definito nella classe \textbf{InputDataset}.
\lstinputlisting{files/In.java}
La classe \textbf{InputDataset} ha come variabili d'istanza (linee 8-11) un ArrayList \textit{buf}, che viene utilizzato per identificare univocamente i vari simboli e una matrice \textit{picture} che viene utilizzata per rappresentare l'immagine di rappresentazione dell'input. Alle linee 18-34 è stato definito il metodo \textbf{loadData()} che viene utilizzato per caricare l'input presente in un file nella matrice \textit{picture} e nell'array \textit{buf}. Nella matrice i simboli sono rappresentati attraverso gli identificativi associati dall'array \textit{buf}. Alle linee 37-39 il metodo \textbf{getToken()} restituisce un simbolo in base all'identificativo ricevuto in input. Il metodo \textbf{setTokenFound()} (linee 42-57) viene utilizzato per inserire un simbolo che è stato letto all'interno dell'array dei simboli visti. Infine abbiamo il metodo \textbf{getNextToken()} linee(75-98) che viene utilizzato per ottenere il simbolo successivo. Questo metodo legge la matrice finchè non trova l'ultimo simbolo letto, una volta trovato seleziona le regole per leggere il simbolo successivo (linee 77-84). Se il simbolo è < allora il simbolo successivo deve essere letto sulle righe successive rispetto all'ultimo simbolo visto, altrimenti se il simbolo è > il simbolo successivo viene letto sulle colonne a destra rispetto all'ultimo simbolo visto. Nel caso in cui l'ultimo simbolo letto si trova sull'ultima riga o colonna della tabella il simbolo successivo verrà scelto leggendo tutta la matrice. Dopodichè si legge la matrice partendo dagli indici calcolati dalle relazioni spaziali (linee 86-92) e si restituisce il primo simbolo non visto. Per stabilire se il simbolo è stato visto o no si usa il metodo \textbf{isViewed()} (linee 60-67) che restituisce \textit{false} se il simbolo è stato visto o \textit{true} se il simbolo non è stato visto.
\subsection{La classe GLLParsingPosizionale}
Per implementare il GLL Parsing sulle espressioni aritmetiche è stata creata la classe GLLParsingPosizionale. Gli insiemi che il parsing utilizza sono sempre gli stessi ma presentano piccole differenze rispetto a quello lineare. Gli elementi dell'insieme \textbf{P} e i descrittori dell'insieme \textbf{R} presentano un ulteriore elemento ed è l'array dei token visti. Infatti l'esecuzione di ogni descrittore deve avere i propri simboli visti e di conseguenza ogni elemento \textbf{P} che viene richiamato dal nodo padre \textit{u} del GSS che va ad aggiungersi all'insieme \textbf{R} deve avere il proprio array dei simboli visti.  Alle funzioni \textit{add()} e \textit{pop()} è stato aggiunto un parametro in input per richiedere l'inserimento dell'array dei simboli visti. Nel metodo \textit{main()} viene istanziato l'oggetto \textit{inputDataset}, di tipo InputDataset, viene chiamato il metodo \textit{loadDataset()} per caricare l'input nella matrice ed infine viene chiamato il metodo \textit{parse()} per iniziare il parsing ed a questo metodo viene passato come parametro \textit{inputDataset}. Il metodo \textit{parse()} è stato implementato seguendo le stesse regole usate per quello lineare ma con delle differenze che sono mostrate qui di seguito.
\begin{lstlisting}
public static String parse(InputDataset ds) {
	ArrayList<String> tokenViews=new ArrayList<String>();
	//...
	while(true){
		switch(etichetta){
		// (.>E>)
		case "L90":
			i = ds.getNextToken(i, ">", tokenViews);
			etichetta = "L10";
			break;
		// F.<hbar<T
		case "L6":
			i = ds.getNextToken(i, "<", tokenViews);
			etichetta = "L50";
			break;
		// F<.hbar<T
		case "L50":
			if (ds.getToken(i).equals("hbar")) {
				ds.setTokenFound(i, tokenViews);
			
				//...
			
				etichetta = "L7";
			} 
			else {
				etichetta = "L0";
			}
			break;
		}
	}
	//...
}
\end{lstlisting}
Le linee 1-10 mostrano come il modo in cui il parser gestisce le relazioni spaziali. Si può notare che vengono gestiti come item all'interno del corpo della produzione, ed ogni volta che lo incontra viene chiamato il metodo \textit{getNextToken()} che prende in input un operatore spaziale, l'utimo simbolo letto e l'array dei simboli visti. Questo metodo restituisce il simbolo successivo da leggere; dopodichè si passa all'item successivo. In corrispondenza della lettura di un input (linee 15-28) viene verificata l'uguaglianza del simbolo corrente che si sta leggendo, se è vera il simbolo letto viene messo nell'array dei simboli visti.
\section{GLL Parsing sui diagrammi di flusso}
Ora illustreremo un altro esempio di implementazione del GLL Parsing su un'altra grammatica posizionale. La seguente grammatica posizionale definisce i diagrammi di flusso.\par
\begin{center}
	\textit{Program $\to$ START link(1,1) Statements link(2,1) END} \par
	\textit{Statements $\to$ Statement link(2,1) Statements $\{$$\$\$$.1=$\$$1.1; $\$\$$.2=$\$$2.2;$\}$} \par
	\textit{Statements $\to$ Statement $\{$$\$\$$.1=$\$$1.1; $\$\$$.2=$\$$1.2;$\}$}\par
	\textit{Statement $\to$ INSTRUCTION $\{$$\$\$$.1=$\$$1.1; $\$\$$.2=$\$$1.2;$\}$}\par
	\textit{Statement $\to$ PREDICATE link(2,1) $\wedge$ link(3,2) Statements $\{$$\$\$$.1=$\$$1.1; $\$\$$.2=$\$$2.2;$\}$}\par
	\textit{Statement $\to$ PREDICATE  link(2,1) $\wedge$ link(1,2) Statements $\{$$\$\$$.1=$\$$1.1; $\$\$$.2=$\$$1.3;$\}$}\par
	\textit{Statement $\to$ PREDICATE  link(2,1) $\wedge$ nolink(1,2) Statements link(3,1)(-1) $\wedge$ link(2,2) Statements $\{$$\$\$$.1=$\$$1.1; $\$\$$.2=$\$$2.2;$\}$}\par
\end{center}
In questa grammatica le relazioni spaziali sono rappresentati dai \textit{link(i,j)} dove \textit{i}, \textit{j} $\geq$ 0. Essi indicano come leggere il simbolo successivo e va interpretato in questo modo: 
\begin{itemize}
	\item Se \textit{link(i,j)} si trova dopo un terminale il simbolo successivo sarà il primo simbolo non letto che ha in comune uno \textit{j-esimo} collegamento con uno \textit{i-esimo} collegamento con il simbolo che è stato appena letto; la lettura di questi simboli è detta \textbf{Driver}.
	\item Se \textit{link(i,j)} è preceduto da un non-terminale il simbolo successivo sarà il primo simbolo che ha in comune lo \textit{i-esimo} collegamento di uno statement con il \textit{j-esimo} collegamento di un simbolo; la lettura di questi simboli è detta \textbf{Tester}.
	\item La seguente condizione \textit{link(i,j) $\wedge$ link(x,y)} va interpretato in questo modo: prima verifichiamo che esistano questi collegamenti e ciò lo facciamo attraverso l'operatore \textit{and} ed usando il modo \textit{Driver} per entrambi i \textit{link}, se è vera calcoliamo il simbolo successivo usando solo \textit{link(i,j)} secondo il modo \textit{Driver}; poi, dopo aver calcolato un non-terminale all'interno di un item della produzione, usiamo \textit{link(x,y)} in modo \textit{Tester} per verificare se l'item appena calcolato restituisce il terminale presente all'interno della produzione. Ciò viene fatto per valutare la correttezza del parsing nel riconoscere i vari simboli.
	\item L'operatore \textit{nolink(i,j)} indica che non deve esistere nessun collegamento tra un simbolo che ha il collegamento \textit{i-esimo} e il simbolo che ha il collegamento \textit{j-esimo}.
	\item Il simbolo (-1) che si trova dopo \textit{link(i,j)} vuol dire che il simbolo successivo deve essere calcolato usando il terminale all'interno della produzione.
\end{itemize} 
Le espressioni \textit{$\{$$\$\$$.1=$\$$1.1; $\$\$$.2=$\$$y.z;$\}$} indicano come costruire gli \textbf{Statement} utilizzati per leggere i simboli succ in modo \textit{Tester}. La dicitura \textit{$\$\$$.1=$\$$1.1} indica che il primo collegamento dello statement che si sta creando corrisponde al primo collegamento del terminale appartenente alla produzione che si sta processando, mentre \textit{$\$\$$.2=$\$$y.z} può avere diverse interpretazioni:
\begin{itemize}
	\item \textit{$\$\$$.2=$\$$2.2}, dove \textit{y=z}; indica che il secondo collegamento dello statement che si sta creando corrisponde al secondo collegamento dell'ultimo statement creato;
	\item \textit{$\$\$$.2=$\$$1.z}, dove y=1; indica che il secondo collegamento dello statement che si sta creando corrisponde al collegamento \textit{z} del terminale appartenente alla produzione che si sta processando.
\end{itemize}
\subsection{Struttura dei simboli e gestione dell'input}
In questa sezione discuteremo la struttura degli statement, dei simboli in input e di come è stato implementata la gestione dell'input. Qui di seguito viene mostrata la classe \textbf{\textit{Token.java}}.
\lstinputlisting{C:/Users/fabio/Documents/GitHub/gll-parsing/src/dataset/Token.java}
La classe \textbf{Token} mostra la struttura di un simbolo. Ha come variabili d'istanza (linee 6-8) una variabile booleana \textit{start}, che serve ad indicare se il primo simbolo da processare; un ArrayList<String> \textit{attachPoints} che contiene i collegamenti che un simbolo ha con altri simboli, ed infine una variabile \textit{type} di tipo String che indica il nome del simbolo. Dalle linee 10-43 vengono dichiarati i costruttori delle classi, i metodi di accesso alle variabili e il metodo \textit{toString()}.\par
Ora presentiamo la classe \textbf{Statement} descritta nel file \textbf{\textit{Statement.java}}
\lstinputlisting{files/St.java}
Questa classe viene utilizzata per definire la struttura degli statement. Ha come variabili d'istanza (linee 5-8): \textit{type1} e \textit{type2}, di tipo \textit{String}, che serve ad indentificare, rispettivamente, a quale simbolo appartiene il primo e il secondo collegamento; poi abbiamo \textit{firstLink} e \textit{secondLink}, di tipo \textit{String} che indicano il primo e il secondo collegamento di uno statement. Dalle linee 11-42 abbiamo i costruttori della classe, i metodi di accesso alle variabili, il metodo per impostare il secondo collegamento allo statement, e il metodo \textit{toString()}.\par
Ora presentiamo la classe \textbf{InputHandler} descritta all'interno del file \textbf{\textit{InputHandler.java}}.
\lstinputlisting{files/InH.java}
La classe InputHandler si occupa di gestire l'input e di calcolare i token successivi. Ha come variabile d'istanza un'array \textit{buf}, che rappresenta l'input da processare. Alle linee 17-29 vi sono i metodi \textit{getFirstToken()} e \textit{getToken()} che vengono utilizzati rispettivamente per ottenere il primo simbolo da far processare al parser e per ottenere il simbolo corrente che il parser deve processare. Il metodo \textit{loadInput()} (linee 31-63) viene utilizzato per caricare l'input, contenuto all'interno di file di tipo json, all'interno dell'array \textit{buf}. Infine (linee 65-129) sono stati implementati i metodi che permettono di  calcolare i token successivi da processare. Il metodo \textit{getTokenDriver()} viene utilizzato per leggere il simbolo successivo secondo il modo \textit{Driver}. Infatti ottiene i collegamenti dal simbolo appena letto e poi vengono iterati tutti i simboli dell'input e alla fine viene restituito l'indice del primo simbolo non visto che ha in comune un collegamento con il simbolo appena letto. Per verificare se il simbolo è stato visto o meno viene utilizzato il metodo \textit{isViewed()}. Il metodo \textit{getTokenTester()} viene utilizzato per calcolare il simbolo successivo secondo il modo \textit{Tester}. Infatti abbiamo che viene selezionato il collegamento dello statement che ci serve per calcolare il simbolo successivo, poi vengono iterati tutti i simboli dell'input e viene restituito il primo simbolo non visto che ha il collegamento in comune con lo statement. Anche qui usiamo il metodo \textit{isViewed()} per verificare se un simbolo è stato visto o no. In alcuni casi abbiamo la necessità di non restituire il primo simbolo non visto e per questo motivo il metodo restituirà sempre \textit{true} se l'array dei simboli visti è \textit{null}. Infine il metodo \textit{setTokenFound()} viene usato per aggiungere un simbolo visto nel'array dei simboli visti.
\subsection{La classe GLLParsingFlowChart}
La classe GLLParsingFlowChart implementa il funzionamento del GLL parsing posizionale sulla grammatica dei diagrammi di flusso. Questo parser usa le stesse strutture dati del parsing posizionale descritto precedentemente ma con delle differenze: gli elementi dell'insieme \textbf{R} e \textbf{P} hanno un ulteriore elemento che è rappresentato dallo stack in cui sono memorizzati gli statement calcolati. Infatti ogni descrittore ed ogni elemento \textbf{P} che viene calcolato deve avere il proprio stack. Di conseguenza ai metodi \textit{add()} e \textit{pop()} è stato aggiunto un parametro che permette di passare lo stack degli statement. Di seguito descriveremo il metodo \textit{parse()} di questa classe tralasciando i dettagli inerenti alla gestione dei non-terminali, terminali e item e sarà trattata soltanto la gestione del calcolo dei simboli successivi e degli statement.
\begin{lstlisting}
	public static String parse(InputHandler buf){
		//token visti
		ArrayList<Integer> tokenViews=new ArrayList<Integer>();
		//...
		//dichirazione indici
		int i=buf.getFirstToken();
		//inizializzo link calcolati
		//...
		Stack<Statement> s=new ArrayStack<Statement>();
		while(true){
			switch(etichetta){
			//...
			//Statement -> INSTRUCTION *{ $$.1 = $1.1; $$.2 = $1.2; }
			case "L11":
				s.push(new Statement(buf.getToken(i).getType(),buf.getToken(i).getAttachPoints().get(0),buf.getToken(i).getType(),buf.getToken(i).getAttachPoints().get(1)));
				etichetta="L12";
				break;
			//...
			
			//Statements *link(2,1) Statements { $$.1 = $1.1; $$.2 = $2.2; }
			case "L5":
				i=buf.getTokenTester(s.top(),2,1,tokenViews);
				if(i>0) {
					etichetta="L6";
				}
				else {
					etichetta="L0";
				}
				break;
			//..
			//Statements link(2,1) Statements { $$.1 = $1.1; $$.2 = $2.2; }*
			case "L8":
				if(s.size()>1) {
					Statement cv=s.pop();
					s.top().setSecondoAttacco(cv.getType2(), cv.getSecondoAttacco());
				}
				etichetta="L0";
				break;
			//...
			//*PREDICATE link(2, 1) ^ nolink(1,2) Statements link(3,1)(-1) ^ link(2,2) Statements   { $$.1 = $1.1; $$.2 = $2.2; }
			case "LSTAT3":
				if(buf.getToken(i).getType().equals("PREDICATE")) {
					buf.setTokenFound(i, tokenViews);
					//...
					etichetta="L17";
				}
				else {
					etichetta="L0";
				}
				break;
			//PREDICATE *link(2, 1) ^ nolink(1,2) Statements link(3,1)(-1) ^ link(2,2) Statements   { $$.1 = $1.1; $$.2 = $2.2; }
			case "L17":
			if((buf.getTokenDriver(i,2,1,tokenViews)>0)&&(!(buf.getTokenDriver(i, 1, 2,tokenViews)>0))) {
				i=buf.getTokenDriver(i,2,1,null);
				etichetta="L18";
			}
			else {
				etichetta="L0";
			}
			break;
			//...
			case "L19":
			i=buf.getTokenTester(s.top(),1,2,null);
			if(i>0) {
				if((buf.getToken(i).getType().equals("PREDICATE"))&&(!(buf.getToken(i).getAttachPoints().get(0).equals(s.top().getSecondoAttacco())))) {
					i=buf.getTokenDriver(i,3,1,null);
					etichetta="L55";
				}
				else {
					s.pop();
					etichetta="L0";
				}
			}
			else {
				s.pop();
				etichetta="L0";
			}
			break;
			case "L55":
				if(i>0) {
					add("L20",cu,i,cn,s.duplica(),duplicaTokenViews(tokenViews));
					add("L5",cu,i,cn,s.duplica(),duplicaTokenViews(tokenViews));
				}
				etichetta="L0";
				break;
			//...
			//PREDICATE link(2, 1) ^ nolink(1,2) Statements link(3,1)(-1) ^ link(2,2) Statements   *{ $$.1 = $1.1; $$.2 = $2.2; }
			case "L21":
				Statement s1=s.pop();
				Statement s2=s.pop();
				i=buf.getTokenTester(s1, 1, 3,null);
				if(i>0) {
					if((s1.getSecondoAttacco().equals(s2.getSecondoAttacco()))&&(buf.getToken(i).getType().equals("PREDICATE"))) {
						s.push(new Statement(buf.getToken(i).getType(),buf.getToken(i).getAttachPoints().get(0),s1.getType2(),s1.getSecondoAttacco()));
						etichetta="L22";
					}
					else {
						etichetta="L0";
					}
				}
				else {
					etichetta="L0";
				}
				break;
				//...
			}
		}
	//...
	}
\end{lstlisting}
Alle linee 2-6 viene dichiarato l'array dei simboli visti, viene calcolato il primo simbolo da calcolare e viene dichiarato lo stack dove memorizzare gli statement. Alle linee 12-15 notiamo la costruzione di uno statement. Ciò avviene usando il primo e il secondo collegamento del simbolo letto e viene inserito nello stack tramite un operazione di \textit{push}. Nelle linee 18-36 abbiamo che la lettura del simbolo successivo attraverso il modo \textit{Tester}. In queste linee la costruzione dello statement avviene in maniera diversa: abbiamo che viene fatto un \textit{pop()} sullo stack se ce ne sono più di uno, e viene impostato al secondo collegamento dello statement in cima allo stack il secondo collegamento dello statement appena estratto. In questo modo vengono rimossi tutti gli statement calcolati precedentemente in maniera tale che alla fine ne rimanga uno soltanto che indica l'inizio e la fine del diagramma di flusso. Alle linee 39-107 discutiamo la gestione degli item delle produzioni. Ne prendiamo di riferimento uno soltanto in quanto le altre si gestiscono allo stesso modo. Viene letto il token corrente, se risulta trovato viene inserito nell'array dei token visti. Poi viene letto il token successivo facendo un \textit{and} tra due link usando il modo \textit{Driver}. Da notare in questo caso l'uso di \textit{null} per segnalare che non è necessario trovare il primo token non visto. Se è vero calcoliamo il token successivo usando il primo \textit{link} e il modo \textit{Driver}. Poi calcoliamo l'item successivo (parte omessa poichè si tratta di gestire un non-terminale e si fa usando le stesse regole). Successivamente verifichiamo se lo statement calcolato sia corretto e lo facciamo tramite il secondo \textit{link} usato nell'and precedente e usando il modo \textit{Tester}. Poi verifichiamo se la lettura del simbolo successivo ha avuto buon fine, se ha successo, verifichiamo che il simbolo calcolato sia quello all'interno della produzione e che siano diversi i collegamenti tra l'ultimo statement calcolato, che si trova in cima allo stack, e il primo collegamento del simbolo della produzione. Se ciò è vero calcoliamo il \textit{link(3,1)(-1)} tramite il modo \textit{Driver}. Fondamentale risulta essere (-1) poichè ci indica di rileggere il simbolo successivo dal simbolo della produzione. Nel caso tutto ciò non fosse vero rimuoviamo lo statement calcolato dallo stack in quanto non risulta essere corretto a rappresentare la produzione che si sta calcolando. Poi notiamo (linee 78-83) che il parser si sdoppia nuovamente dove un parser va in avanti a calcolare lo statement successivo della produzione, mentre l'altro va a calcolare se esistono nuovi statement allo statement calcolato precedentemente. Dopo aver calcolato tutti gli statement calcolati dalla produzione li estraiamo dallo stack e tramite un \textit{link()} in modo \textit{Tester} otteniamo il simbolo della produzione. Verifichiamo prima se la ricerca del simbolo successivo ha avuto buon fine, poi verifichiamo se il secondo collegamento del primo statement calcolato sia uguale al secondo collegamento del secondo statement calcolato ed infine verifichiamo che il simbolo calcolato appartenga al simbolo della produzione. Se è vero tramite un operazione di push inseriamo lo statement finale della produzione e si va all'item successivo. Se tutto ciò non risulta vero si va al caso base L0. In maniera simile vengono gestite le altre produzioni ma avendo un solo statement da calcolare non effettuiamo nessun nuovo sdoppiamento ma semplicemente usiamo il secondo \textit{link()} dell'and in modo \textit{Tester} e poi si verifica che il simbolo appartenga alla produzione. Ciò serve per verificare che lo statement calcolato sia corretto con la produzione che si sta processando. Se ciò è vero lo estraiamo dallo stack, tramite un operazione di \textit{pop()}, e inseriamo lo statement finale nella produzione nello stack.