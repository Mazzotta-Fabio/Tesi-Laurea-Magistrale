\chapter{Implementazione del GLL Posizionale}
\section{Introduzione}
Nel capitolo seguente vedremo come è stato implementato il GLL parsing sulle grammatiche posizionali. Descriveremo come sono state gestite le relazioni spaziali e come è stato applicato su due grammatiche posizionali. Per il resto la gestione dei non terminali e degli item delle varie produzioni sono state gestite allo stesso modo che vale per il GLL parsing lineare.
\section{GLL Parsing Posizionale}
In questo paragrafo discutiamo di come è stato implementato e gestito il GLL Parsing sulla grammmatica posizionale delle espressioni aritmetiche. La grammatica è la seguente:
\begin{align}\label{gramPos1}
E & \to T > + > E \mid  T \notag \\
T & \to F < hbar < T \mid F \notag \\
F & \to ( > E > ) \mid id 
\end{align}
Il simbolo \textbf{>} e \textbf{<} sono relazioni spaziali che indicano rispettivamente di spostarsi in orizzontale e in verticale per leggere il simbolo successivo.
\subsection{Gestione dell'input}
La gestione dell'input è stato definito nella classe \textbf{InputDataset}.
\lstinputlisting{files/In.java}
La classe \textbf{InputDataset} ha come variabili d'istanza (linee 8-11) un ArrayList \textit{buf}, che viene utilizzato per identificare univocamente i vari simboli e una matrice \textit{picture} che viene utilizzata per rappresentare l'immagine di rappresentazione dell'input. Alle linee 18-34 è stato definito il metodo \textbf{loadData()} che viene utilizzato per caricare l'input presente in un file nella matrice \textit{picture} e nell'array \textit{buf}. Nella matrice i simboli sono rappresentati attraverso gli identificativi associati dall'array \textit{buf}. Alle linee 37-39 il metodo \textbf{getToken()} restituisce un simbolo in base all'identificativo ricevuto in input. Il metodo \textbf{setTokenFound()} (linee 42-57) viene utilizzato per inserire un simbolo che è stato letto all'interno dell'array dei simboli visti. Infine abbiamo il metodo \textbf{getNextToken()} linee(75-98) che viene utilizzato per ottenere il simbolo successivo. Questo metodo legge la matrice finchè non trova l'ultimo simbolo letto, una volta trovato seleziona le regole per leggere il simbolo successivo (linee 77-84). Se il simbolo è < allora il simbolo successivo deve essere letto sulle righe successive rispetto all'ultimo simbolo visto, altrimenti se il simbolo è > il simbolo successivo viene letto sulle colonne a destra rispetto all'ultimo simbolo visto. Nel caso in cui l'ultimo simbolo letto si trova sull'ultima riga o colonna della tabella il simbolo successivo verrà scelto leggendo tutta la matrice. Dopodichè si legge la matrice partendo dagli indici calcolati dalle relazioni spaziali (linee 86-92) e si restituisce il primo simbolo non visto. Per stabilire se il simbolo è stato visto o no si usa il metodo \textbf{isViewed()} (linee 60-67) che restituisce \textit{false} se il simbolo è stato visto o \textit{true} se il simbolo non è stato visto.
\subsection{La classe GLLParsingPosizionale}
Per implementare il GLL Parsing sulle espressioni aritmetiche è stata creata la classe GLLParsingPosizionale. Gli insiemi che il parsing utilizza sono sempre gli stessi ma presentano piccole differenze rispetto a quello lineare. Gli elementi dell'insieme \textbf{P} e i descrittori dell'insieme \textbf{R} presentano un ulteriore elemento ed è l'array dei token visti. Infatti l'esecuzione di ogni descrittore deve avere i propri simboli visti e di conseguenza ogni elemento \textbf{P} che viene richiamato dal nodo padre \textit{u} del GSS che va ad aggiungersi all'insieme \textbf{R} deve avere il proprio array dei simboli visti.  Alle funzioni \textit{add()} e \textit{pop()} è stato aggiunto un parametro in input per richiedere l'inserimento dell'array dei simboli visti. Nel metodo \textit{main()} viene istanziato l'oggetto \textit{inputDataset}, di tipo InputDataset, viene chiamato il metodo \textit{loadDataset()} per caricare l'input nella matrice ed infine viene chiamato il metodo \textit{parse()} per iniziare il parsing ed a questo metodo viene passato come parametro \textit{inputDataset}. Il metodo \textit{parse()} è stato implementato seguendo le stesse regole usate per quello lineare ma con delle differenze che sono mostrate qui di seguito.
\begin{lstlisting}
public static String parse(InputDataset ds) {
	ArrayList<String> tokenViews=new ArrayList<String>();
	//...
	while(true){
		switch(etichetta){
		// (.>E>)
		case "L90":
			i = ds.getNextToken(i, ">", tokenViews);
			etichetta = "L10";
			break;
		// F.<hbar<T
		case "L6":
			i = ds.getNextToken(i, "<", tokenViews);
			etichetta = "L50";
			break;
		// F<.hbar<T
		case "L50":
			if (ds.getToken(i).equals("hbar")) {
				ds.setTokenFound(i, tokenViews);
			
				//...
			
				etichetta = "L7";
			} 
			else {
				etichetta = "L0";
			}
			break;
		}
	}
	//...
}
\end{lstlisting}
Le linee 1-10 mostrano come il modo in cui il parser gestisce le relazioni spaziali. Si può notare che vengono gestiti come item all'interno del corpo della produzione, ed ogni volta che lo incontra viene chiamato il metodo \textit{getNextToken()} che prende in input un operatore spaziale, l'utimo simbolo letto e l'array dei simboli visti. Questo metodo restituisce il simbolo successivo da leggere; dopodichè si passa all'item successivo. In corrispondenza della lettura di un input (linee 15-28) viene verificata l'uguaglianza del simbolo corrente che si sta leggendo, se è vera il simbolo letto viene messo nell'array dei simboli visti.
