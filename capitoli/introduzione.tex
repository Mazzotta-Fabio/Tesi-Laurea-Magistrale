\chapter{Introduzione}
\section{Obiettivi}
Questa tesi di laurea descrive il funzionamento e l'implementazione del parsing \textbf{Generalizzato LL (GLL)} sui linguaggi non lineari. Il parsing GLL è un algoritmo di parsing top down che viene utilizzato per gestire tutte le grammatiche context-free che sono ambigue e ricorsive a sinistra. La caratteristica principale di questo algoritmo è che risulta essere un parser a \textbf{discesa ricorsiva} e ciò permette di avere il controllo del flusso sulle strutture della grammatica e risultano semplici da implementare e semplici da testare passo dopo passo attraverso il debugger. Questo parser è stato utilizzato per riconoscere linguaggi non lineari (bidimensionali) generati da grammatiche posizionali, ossia generalizzazioni di grammatiche context-free. La tesi è divisa in tre parti. Nella prima parte si cerca di illustrare come funziona il parsing LL, che rappresenta la base del parsing GLL, e i suoi limiti. Successivamente si  discuterà come estendere il parsing LL attraverso il parsing GLL, illustrandone i principi e le strutture dati che utilizza. Ciò viene descritto rispettivamente nel secondo e terzo capitolo. Nella seconda parte si analizzaranno le grammatiche posizionali. Questo argomento sarà trattato nel quarto capitolo. Nell'ultima parte si parlerà l'implementazione del parsing GLL applicato ad una grammatica posizionale. In particolare nel quinto capitolo si descriverà le varie  componenti software del parsing GLL, nel sesto capitolo si illustrerà come viene applicato il software del parsing GLL ad una grammatica posizionale e nel settimo capitolo si parlerà del tool utilizzato per testare il software del parsing GLL. Infine nell'ottavo capitolo si discuteranno i risultati ottenuti e gli sviluppi futuri.
