\chapter{Introduzione}
\section{Introduzione}
Questa tesi di laurea descrive il funzionamento e l'implementazione del parsing \textbf{Generalizzato LL (GLL)}. Il parsing GLL è un algoritmo di parsing top down che viene utilizzato per gestire tutte le grammatiche context-free comprese quelle che sono ambigue e ricorsive a sinistra e a destra. La caratteristica principale di questo algoritmo è che risulta essere un parser a \textbf{discesa ricorsiva} e ciò permette di avere il controllo del flusso sulla struttura della grammatica e di conseguenza risulta semplice da implementare e da testare. L'obiettivo da raggiungere sarà quello di far riconoscere al GLL parsing i linguaggi non lineari (bidimensionali) prodotti dalle grammatiche posizionali. La tesi è divisa in tre parti. Nella prima parte si illustreranno i principi su cui si basa il funzionamento del GLL parsing e verrà descritto il parsing top down, che rappresenta la base di funzionamento del GLL parsing, e i suoi limiti. Successivamente si introdurrà il GLL parsing, illustrandone i principi e i meccanismi che usa per superare i limiti dei parser top down tradizionali, le strutture dati che utilizza e il risultato ottenuto dalla sua computazione. Ciò viene descritto rispettivamente nel secondo e terzo capitolo. Nella seconda parte verrà descritto il funzionamento del GLL parsing che opera su grammatiche posizionali per riconoscere i linguaggi non lineari. Questo argomento sarà trattato nel quarto capitolo. Infine, nell'ultima parte si parlerà dell'implementazione del GLL parsing applicato ad una grammatica context-free e ad una grammatica posizionale. Ciò verrà descritto nel quinto e sesto capitolo. La tesi si conclude al settimo capitolo in cui vengono descritti i risultati e gli obiettivi raggiunti.