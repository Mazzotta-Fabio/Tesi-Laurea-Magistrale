\chapter{Implementazione del GLL parsing}
\section{Introduzione}
In questo capitolo illustreremo come è stato implementato il GLL Parsing. Parleremo delle componenti software che lo compongono ed illustremo come è stato utilizzato per un linguaggio lineare. Inoltre tratteremo anche della costruzione del SPPF.
\section{Le componenti del sistema}
Per implementare il GLL parsing si è cercato di creare tutte le componenti usate dall'algoritmo che sono state accennate al capitolo 3. Queste componenti sono:
\begin{itemize}
	\item \textbf{ElementoU}: è una componente che rappresenta un elemento memorizzato dall'insieme \textbf{U},
	\item \textbf{ElementoP}: si occupa di gestire gli elementi dell'insieme \textbf{P},
	\item \textbf{DescrittoreR}: questa componente rappresenta un descrittore che viene memorizzato dall'insieme \textbf{R} 
	\item \textbf{GLLParsing}: è una componente che gestisce la computazione del GLL parsing.
\end{itemize}
Di seguito mostriamo il class diagram del sistema e nei paragrafi successivi tratteremo nei dettagli ogni componente del sistema.
\begin{figure}[h]
	\flushleft
	\includegraphics[width=470pt,height=290pt]{C:/Users/fabio/Documents/GitHub/GLLParsing/grafico.jpg}
	\caption{\textit{Class diagram del software del GLL Parsing}}
\end{figure}
\section{Le classi degli insiemi R, P e U}
In questo paragrafo discutiamo delle classi che rappresentano gli elementi memorizzati dagli insiemi \textbf{P}, \textbf{R} e \textbf{U}. Qui di seguito presentiamo la classe \textbf{\textit{ElementoU}}
\lstinputlisting{C:/Users/fabio/Documents/GitHub/gll-parsing/src/gllparsinglineare/ElementoU.java}
Questa classe definisce gli elementi appartenenti all'insieme \textbf{U}. Infatti questa classe ha come variabili d'istanza \textit{etichetta}, di tipo \textit{String}, che rappresenta l'identificativo usato per spostarsi sui vari item delle produzioni, ed un nodo \textit{u}, di tipo \textit{Vertex<String>}, che mantiene traccia del nodo del GSS che il parser sta processando. Presenta un costruttore per inizializzare le variabili d'istanza al momento della creazione dell'oggetto (linee 8-10), dei metodi d'accesso alle variabili d'istanza (linee 13-19) ed il metodo \textit{ toString()} per descrivere lo stato dell'oggetto (linee 21-22).\\
Ora descriviamo la classe \textbf{\textit{ElementoP}}
\lstinputlisting{C:/Users/fabio/Documents/GitHub/gll-parsing/src/gllparsinglineare/ElementoP.java}
Questa classe rappresenta un elemento dell'insieme \textbf{P}. Ha come variabili d'istanza \textit{u} di tipo \textit{Vertex<String>}, che rappresenta un nodo del GSS, un intero \textit{k}, che rappresenta la posizione di un simbolo all'interno dell'input, ed una variabile d'istanza \textit{z}, di tipo \textit{Vertex<IdNodoSppf>} che rappresenta un nodo del sppf (linee 7-9). Possiede un costruttore  per inizializzare le variabili d'istanza al momento della creazione dell'oggetto (linee 11-15), dei metodi d'accesso alle variabili d'istanza (linee 17-27) ed il metodo \textit{toString()} che serve per descrivere lo stato dell'oggetto (linee 35-37). La presenza del nodo del sppf è importante perchè ogni volta che viene fatto un \textit{pop()} il parser deve riprendere la costruzione dell'sppf ripartendo dall'ultimo nodo inserito.\\
Ora presentiamo la classe \textbf{\textit{DescrittoreR.java}}.
\lstinputlisting{C:/Users/fabio/Documents/GitHub/gll-parsing/src/gllparsinglineare/DescrittoreR.java}
Questa classe rappresenta un descrittore che viene memorizzato nell'insieme \textbf{R}. Ha come varibili d'istanza un'\textit{etichetta}, di tipo \textit{String}, che indica l'item che deve essere computato, un nodo \textit{u}, che indica il nodo del GSS, un intero \textit{i}, che indica un simbolo della stringa in ingresso che si vuole trovare ed un nodo \textit{w}, che indica un nodo del SPPF su cui il parser deve fare la costruzione (linee 7-10). Possiede un costruttore per inizializzare le variabili al momento della creazioe dell'oggetto (linee 12-17), dei metodi d'accesso alle variabili e il metodo \textit{toString()} per descrivere lo stato di un oggetto
\section{La classe GLL Parser}
Inn questo paragrafo descriviamo la classe \textbf{\textit{GLLParsing}} che definisce l'implementazione del GLL Parsing. Illustremo in più parti le varie operazioni che svolge. Il parsing viene eseguito sulla grammatica \ref{gram3} 
\begin{lstlisting}
package gllparsing;

import graph.*;
import java.io.*;
import java.util.*;
	
/*
	* GRAMMATICA
	* S->ASd
	* S->BS
	* S->epsilon
	* A->a
	* A->c
	* B->a
	* B->b
*/
public class GLLParsing {
	//gss
	private static Graph<String> gss;
	//insieme r e u che sono gli insiemi usati per registrare le scelte del non determinismo
	private static ArrayList<ElementoU> u;
	private static ArrayList<DescrittoreR>r;
	//insieme p
	private static ArrayList<ElementoP>p;
	//sppf
	private static Graph<IdNodoSppf> sppf;
	
	public static void main(String []args) {
		File f=new File("file.txt");
		Scanner buffer;
		if(f.exists()){
			try{
				FileReader in=new FileReader(f);
				buffer=new Scanner(in);
				char[] buf;
				while(buffer.hasNextLine()){
					String line=buffer.nextLine();
					buf=line.toCharArray();
					gss=new Graph<String>();
					sppf=new Graph<IdNodoSppf>();
					r=new ArrayList<DescrittoreR>();
					u=new ArrayList<ElementoU>();
					p=new ArrayList<ElementoP>();
					String esito=parse(buf);
					System.out.println(esito);
				}	
			}
			catch(Exception e){
				e.printStackTrace();
			}
		}
		else{
			System.out.println("File not Found");
		} 
	}	
\end{lstlisting}
Iniziamo a commentare le parti sostanziali. Alle linee 19-26 vengono dichiarate le variabili d'istanza che sono: il \textit{GSS} ed \textit{SPPF}, di tipo \textit{Graph}; l'insieme \textit{R}, un \textit{ArrayList} che contiene gli elementi di tipo \textit{DescrittoreR}; l'insieme \textit{U}, un \textit{ArrayList} che ha elementi di tipo \textit{ElementoU} e l'insieme \textit{P}, un \textit{ArrayList} che possiede gli elementi di tipo \textit{ElementoP}. Poi viene definito il metodo \textit{main()} (linee 28-55) che avvia il parser. La stringa da analizzare si trova all'interno di un file e viene inserita nell'array \textit{buf}, di tipo \textit{char}. Vengono istanziate le variabili d'istanza e viene chiamato il metodo \textit{parse()}, dove passiamo \textit{buf}, per fare il parsing della stringa. Alla fine viene stampato l'esito del parsing.\\
Ora analizziamo le funzioni fondamentali.
\begin{lstlisting}
	private static void add(String etichetta, Vertex<String> nu,int j,Vertex<IdNodoSppf>cn){
		u.add(new ElementoU(etichetta,nu));
		r.add(new DescrittoreR(etichetta,nu,j,cn));
	}
\end{lstlisting}
Il metodo \textit{add()} ha come parametri l'\textit{etichetta}, l'identificativo di un item che deve essere computato, un intero \textit{j}, la posizione del simbolo della stringa che si vuole trovare, un nodo \textit{nu} che indica un nodo del GSS, ed un nodo \textit{cn}, un nodo dell'SPPF. Il metodo aggiunge gli elementi agli insiemi \textbf{R} ed \textbf{U}.
\begin{lstlisting}
	private static Vertex<String> create(String etichetta,Vertex<String> u,int j,Vertex<IdNodoSppf>cn){
		//creazione del nodo
		String nomeNodo="Ls"+j+etichetta;
		Vertex<String> v=null;
		Iterator<Vertex<String>> iteratorNodes=gss.vertices();
		while(iteratorNodes.hasNext()){
			Vertex<String> last=iteratorNodes.next();
			if(nomeNodo.equals(last.element())){
				v=last;
			}
		}
		if(v==null){
			v=gss.insertVertex(nomeNodo);
		}
		//controlliamo arco tra v ed u
		Iterator<Edge<String>> eset=gss.edges();
		boolean flag=true;
		while(eset.hasNext()){
			Edge<String> ed=eset.next();
			Vertex<String> u1=ed.getStartVertex();
			Vertex<String> u2=ed.getEndVertex();
			if((u1.element().equals(v.element()))&&(u2.element().equals(u.element()))){
				flag=false;
			}
		}
		if((flag)&&(!(v.element().equals(u.element()))))){
			gss.insertDirectedEdge(v, u, "");
			for(ElementoP elp:p){
				if(elp.getU().element().equals(v.element())){
					add(etichetta,u,elp.getK(),elp.getZ());
				}
			}
		}
		return v;
	}
\end{lstlisting}
Il metodo \textit{create()} prende come parametri un \textit{etichetta}, di tipo \textit{String}, un nodo \textit{u}, che rappresenta un nodo del GSS, un intero \textit{j} che indica la posizione del simbolo della stringa in input che si vuole trovare ed un nodo \textit{cn} che rappresenta un nodo dell'SPPF. Alla linea 3 definiamo il nome del nuovo nodo del gss. La rappresentazione del nodo \textit{L}$^{j}$ descritta al paragrafo \ref{par1} avviene usando questa notazione: \textit{Ls}\textit{j}\textit{etichetta}. Si controlla (linee 4-14) se esiste un nodo \textit{v} nel GSS che possieda il nome creato. Se non esiste lo aggiungiamo al GSS. Alle linee 16-34, si controlla se esiste un arco tra \textit{u} e \textit{v} nel GSS. Se non esiste lo aggiungiamo e di conseguenza per ogni nodo \textit{v} presente in \textbf{P} chiama la funzione \textit{add()}, passandogli \textit{etichetta}, il nodo \textit{u}, l'intero \textit{k} e il nodo dell'SPPF \textit{z} dell'elemento \textbf{P}.
\begin{lstlisting}
	private static void pop(Vertex<String> u,int j,Vertex<String> u0,Vertex<IdNodoSppf>cn){
		//if u diverso da u0
		if(!(u.element().equals(u0.element()))){
			//mettiamo elemento u,j a p
			p.add(new ElementoP(u,j,cn));
			Iterator<Edge<String>> eset=gss.edges();
			//per ogni figlio v di aggiungi lu,v,j ad r e u
			while(eset.hasNext()){
				Edge<String> ed=eset.next();
				Vertex<String>u1=ed.getStartVertex();
				Vertex<String> v1=ed.getEndVertex();
				if(u.element().equals(u1.element())){
					add(u1.element().substring(3),v1,j,cn);
				}
			}
		}
	}
\end{lstlisting}
Il metodo \textit{pop()} ha come parametri un nodo del GSS \textit{u}, un intero \textit{j}, che indica una posizione del simbolo della stringa in input da trovare, il nodo \textit{u0}, che rapprsenta il nodo del GSS $\$$ ed il nodo dell'SPPF \textit{cn}. Il metodo controlla se il nodo \textit{u} è diverso da \textit{u0}. Se lo è aggiunge \textit{u}, \textit{j} e \textit{cn} nell'insieme \textbf{P}. Poi per ogni nodo \textit{u} nel GSS chiamiamo il metodo \textit{add()} e gli passiamo come parametri l'\textit{etichetta} (che sarebbe parte \textit{\textbf{etichetta}} del nome del nodo \textit{u}), il nodo figlio \textit{v} di \textit{u}, l'intero \textit{j} e il nodo \textit{cn}  del SPPF.
\begin{lstlisting}
	//controlla il simbolo buffer corrente di un non terminale 
	private static boolean test(char x,String nonTerm,String item){
		if((first(x,item))||(first('$',item)&&(follow(x,nonTerm)))){
			return true;
		}
		else{
			return false;
		}
	}
	
	private static Vertex<IdNodoSppf> getNodeT(String simbolo,String item,Vertex<IdNodoSppf>cn){
		Iterator<Edge<IdNodoSppf>>it=sppf.edges();
		String nonTerm=item.substring(0,1);
		boolean flag=true;
		while(it.hasNext()) {
			Edge<IdNodoSppf>e=it.next();
			Vertex<IdNodoSppf> v1=e.getStartVertex();
			Vertex<IdNodoSppf> v2=e.getEndVertex();
			if((cn.element().getId()==v1.element().getId())&&(item.equals(v2.element().getItem()))) {
				flag=false;
			}
		}
		if((flag)&&(cn.element().getNomeNodo().equals(nonTerm))) {
			Vertex<IdNodoSppf> v=sppf.insertVertex(new IdNodoSppf(simbolo,item));
			v.element().setId(v.hashCode());
			sppf.insertDirectedEdge(cn, v, null);
			return v;
		}
		return cn;
	}
	
	private static Vertex<IdNodoSppf> getNodeP(Vertex<IdNodoSppf>cn){
		Iterator<Edge<IdNodoSppf>>it=sppf.edges();
		while(it.hasNext()) {
			Edge<IdNodoSppf>e=it.next();
			Vertex<IdNodoSppf> v1=e.getStartVertex();
			Vertex<IdNodoSppf> v2=e.getEndVertex();
			if(v2.element().getId()==cn.element().getId()) {
				return v1;
			}
		}
		return cn;
	}
	
\end{lstlisting}
Infine abbiamo i metodi \textit{test()}, \textit{getNodeT()}, \textit{getNodeP()}. Il metodo \textit{test()} prende come parametri \textit{x}, un simbolo della stringa in input, \textit{A}, un non-terminale di tipo \textit{String} ed \textit{item}, di tipo \textit{String}. Fa le stesse operazioni accennate al paragrafo \ref{par}. Il metodo \textit{getNodeT()} prende come parametro \textit{nomeNodo} che indica il nome del nodo da inserire nell' SPPF, il nodo \textit{cn},e un \textit{item}. Questo metodo inserisce l'arco tra il nodo \textit{cn} e il nodo \textit{v} di nome \textit{nomeNodo} appartenente alla produzione inidicato dall'\textit{item}. Per identificare univocamente i nodi dell'SPPF settiamo un idenficativo con l'\textit{hashcode} dell'oggetto. Il metodo \textit{getNodeP()} prende come parametro un nodo \textit{cn} dell'SPPF. Questo metodo restituisce il nodo padre \textit{u} che ha per figlio il nodo \textit{cn}.\\
Ora analizziamo il metodo \textit{parse()};
\lstinputlisting{files/Prova.java}
Il metodo \textit{parse()} prende come parametro un array \textit{buf} che rappresenta la stringa in input su cui fare il parsing. Alle linee 3-13 vengono inizializzate le variabili usate dal parsing durante la computazione. Si inizializza l'indice \textit{i} a 0 per posizionarlo sul primo elemento della stringa, inseriamo un arco tra i nodi $\$$ e \textit{Ls0L0} nel GSS ed inseriamo il simbolo iniziale nell'SPPF. I \textbf{goto} presentati nell'algoritmo del paragrafo \ref{algoritmoIntero} sono implementati con uno \textbf{\textit{switch }}all'interno di un ciclo \textbf{\textit{while}} infinito. I \textbf{\textit{case}} dello \textbf{\textit{switch}} assumono i possibili valori che può avere un etichetta. Un etichetta rappresenta un item o un non-terminale da gestire. Le linee 19-24 viene gestito un non terminale non LL(1) che presenta dei conflitti. Abbiamo che se il metodo \textit{test()} ha successo, viene chiamato il metodo \textit{add()}, e ciò permette applicare il non determinismo inserendo tutte le possibili sostituzioni nell'insieme \textbf{R} ed \textbf{U}. Alle linee 26-34 vengono gestiti i non-terminali LL(1), qui non viene applicato il non determinismo e di conseguenza questo ci permette di sostituire direttamente il non-terminale settando l'etichetta dell'item opportuno. Ora analizziamo le linee 35-72. In queste linee abbiamo la gestione degli item per un corpo di una produzione. Alle linee 36-56 abbiamo la gestione di un non-terminale all'interno dell'item. Viene invocato il metodo \textit{test()}, se ha successo facciamo varie operazioni. La prima operazione che facciamo è creare un nodo nel GSS con l'etichetta dell'item successivo (metodo \textit{create()}), in questo modo quando verrà sostituito il simbolo puntato da questo item, sarà possibile continuare la computazione all'item successivo. Poi aggiungiamo un nodo al SPPF etichettato con il non-terminale in cui si trova il punto nell'item e lo stesso item (metodo \textit{getNodeT()}). Infine settiamo l'etichetta con il valore che ci permette di sostituire il non-terminale dell'item che si sta analizzando. Nel caso in cui il metodo \textit{test()} fallisce torniamo al caso base \textit{L0}. Le linee 47-56 fanno le stesse operazioni però prima di invocare il metodo \textit{getNodeT()} viene invocato il metodo \textit{getNodeP()} per richiamare l'ultimo nodo appartenente all'item che si sta processando. Nelle linee 58-66 si verifica se è stato trovato un simbolo della stringa in input. Se è vero incrementiamo l'indice \textit{i} in maniera tale da farlo passare al simbolo successivo. Poi viene settata l'etichetta all'item successivo. Se la verifica fallisce viene settata l'etichetta del caso base \textit{L0}. Le linee 68-72 gestiscono il completamento di una produzione, ossia il punto dell'item è alla fine del corpo della produzione. Viene chiamato il metodo \textit{getNodeP()}, il metodo \textit{pop()} e l'etichetta viene settata l'etichetta al caso base \textit{L0}. Alle linee 101-107 viene gestita una produzione che ha corpo $\epsilon$. Viene chiamato il metodo \textit{getNodeT()} per aggiungere un nodo all'SPPF, poi viene chiamato il metodo \textit{getNodeP()} per recuperare il nodo padre del nodo aggiunto precedentemente, poi chiamiamo il metodo \textit{pop()} e settiamo l'etichetta al caso base \textit{L0}. Osserviamo che il metodo \textit{getNodeP()} va inserito sempre dal secondo item della produzione che si sta processando. Nel caso in cui abbiamo una produzione con corpo $\epsilon$ vanno aggiunti entrambi, prima \textit{getNodeT()} e poi \textit{getNodeP()}. Alle linee 170-193 è presente la gestione del caso base \textit{L0}. Si verifica se l'insieme \textbf{R} è pieno. Se lo è viene estratto il primo descrittore da \textbf{R} e vengono impostate le variabili \textit{i}, \textit{etichetta}, il nodo del GSS e il nodo dell'SPPF per calcolare una scelta applicata dal non determinismo. Dopodichè viene rimosso da \textbf{R}. Nel caso in cui \textbf{R} è vuoto, si verifica se l'insieme \textbf{U} è vuoto. Se lo è il parsing non ha avuto successo e termina. Altrimenti se l'ultimo descrittore di \textbf{U} ha come etichetta \textit{L0} e il nodo del GSS è $\$$ il parsing termina con successo, se non lo è, termina con insuccesso.