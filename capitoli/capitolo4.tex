\definecolor{javapurple}{rgb}{0.5,0,0.35} % keywords
\definecolor{javadocblue}{rgb}{0.25,0.35,0.75} % javadoc
\definecolor{javagreen}{rgb}{0.25,0.5,0.35} % comments
\definecolor{backcolour}{rgb}{0.95,0.95,0.92}%background
\lstset{language=Java,
	aboveskip=3mm,
	belowskip=3mm,
	columns=flexible,
	frame=single,%tb
	keywordstyle=\color{javapurple}\bfseries,
	stringstyle=\color{blue},
	commentstyle=\color{javagreen},
	morecomment=[s][\color{javadocblue}]{/**}{*/},
	numbers=left,%none
	numberstyle=\color{black},
	stepnumber=1,
	numbersep=20pt,
	tabsize=3,
	breaklines=true,
	showspaces=false,
	showstringspaces=false,
	backgroundcolor=\color{backcolour},
	basicstyle=\ttfamily\footnotesize}
\chapter{Implementazione del GLL parsing}
\section{Introduzione}
In questo capitolo illustreremo come è stato implementato il GLL Parsing. Parleremo delle componenti software che lo compongono ed illustremo come è stato utilizzato per un linguaggio lineare. Inoltre tratteremo anche della costruzione del sppf.
\section{Le componenti del sistema}
Per realizzare il GLL parsing si è cercato di creare tutte le componenti che utilizza l'algoritmo durante la sua computazione e risultano essere:
\begin{itemize}
	\item \textbf{ElementoU}: è una componente che rappresenta un elemento memorizzato dall'insieme \textbf{U},
	\item \textbf{ElementoP}: si occupa di gestire gli elementi dell'insieme \textbf{P},
	\item \textbf{DescrittoreR}: questa componente rappresenta un descrittore che viene memorizzato dall'insieme \textbf{R} 
	\item \textbf{GLLParsing}: è una componente che gestisce la computazione del GLL parsing.
\end{itemize}
Di seguito mostriamo il class diagram del sistema e nei paragrafi successivi tratteremo nei dettagli ogni componente del sistema.
\begin{figure}[h]
	\flushleft
	\includegraphics[width=450pt,height=280pt]{C:/Users/fabio/Documents/GitHub/GLLParsing/grafico.jpg}
	\caption{\textit{Class diagram del software del GLL Parsing}}
\end{figure}
\section{Le classi degli insiemi R, P e U}
In questo paragrafo discutiamo delle classi che rappresentano gli elementi memorizzati dagli insiemi \textbf{P}, \textbf{R} e \textbf{U}. Qui di seguito presentiamo la classe \textbf{\textit{ElementoU.java}}
\lstinputlisting{C:/Users/fabio/Documents/GitHub/GLLParsing/src/gllparsing/ElementoU.java}
Questa classe definisce gli elementi appartenenti all'insieme \textbf{U}. Infatti questa classe ha come variabili d'istanza \textit{etichetta}, di tipo \textit{String}, ed un nodo \textit{u}, di tipo \textit{Vertex}, che mantiene traccia del nodo del GSS che il parser sta processando. Presenta un costruttore per inizializzare le variabili d'istanza al momento della creazione dell'oggetto (linee 8-10), dei metodi d'accesso alle variabili d'istanza (linee 13-19) ed il metodo \textit{ toString()} per descrivere lo stato dell'oggetto (linee 21-22).\\
Ora descriviamo la classe \textbf{\textit{ElementoU.java}}
\lstinputlisting{C:/Users/fabio/Documents/GitHub/GLLParsing/src/gllparsing/ElementoP.java}
\lstinputlisting{C:/Users/fabio/Documents/GitHub/GLLParsing/src/gllparsing/DescrittoreR.java}
\section{La classe GLL Parser}
mettere la classe con il main che indica il gll parsing
\lstinputlisting{C:/Users/fabio/Documents/GitHub/GLLParsing.java}