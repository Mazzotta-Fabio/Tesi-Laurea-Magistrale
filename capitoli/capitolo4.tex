\chapter{Implementazione del GLL parsing}
\section{Introduzione}
In questo capitolo illustreremo come è stato implementato il GLL Parsing. Parleremo delle componenti software che lo compongono ed illustremo come è stato utilizzato per un linguaggio lineare. Inoltre tratteremo anche della costruzione del sppf.
\section{Le componenti del sistema}
Per realizzare il GLL parsing si è cercato di creare tutte le componenti che utilizza l'algoritmo durante la sua computazione e risultano essere:
\begin{itemize}
	\item \textbf{ElementoU}: è una componente che rappresenta un elemento memorizzato dall'insieme \textbf{U},
	\item \textbf{ElementoP}: si occupa di gestire gli elementi dell'insieme \textbf{P},
	\item \textbf{DescrittoreR}: questa componente rappresenta un descrittore che viene memorizzato dall'insieme \textbf{R} 
	\item \textbf{GLLParsing}: è una componente che gestisce la computazione del GLL parsing.
\end{itemize}
Di seguito mostriamo il class diagram del sistema e nei paragrafi successivi tratteremo nei dettagli ogni componente del sistema.
\begin{figure}[h]
	\flushleft
	\includegraphics[width=450pt,height=280pt]{C:/Users/fabio/Documents/GitHub/GLLParsing/grafico.jpg}
	\caption{\textit{Class diagram del software del GLL Parsing}}
\end{figure}
\section{Le classi degli insiemi R, P e U}
In questo paragrafo discutiamo delle classi che rappresentano gli elementi memorizzati dagli insiemi \textbf{P}, \textbf{R} e \textbf{U}. Qui di seguito presentiamo la classe \textbf{\textit{ElementoU.java}}
\lstinputlisting{C:/Users/fabio/Documents/GitHub/GLLParsing/src/gllparsing/ElementoU.java}
Questa classe definisce gli elementi appartenenti all'insieme \textbf{U}. Infatti questa classe ha come variabili d'istanza \textit{etichetta}, di tipo \textit{String}, ed un nodo \textit{u}, di tipo \textit{Vertex}, che mantiene traccia del nodo del GSS che il parser sta processando. Presenta un costruttore per inizializzare le variabili d'istanza al momento della creazione dell'oggetto (linee 8-10), dei metodi d'accesso alle variabili d'istanza (linee 13-19) ed il metodo \textit{ toString()} per descrivere lo stato dell'oggetto (linee 21-22).\\
Ora descriviamo la classe \textbf{\textit{ElementoP.java}}
\lstinputlisting{C:/Users/fabio/Documents/GitHub/GLLParsing/src/gllparsing/ElementoP.java}
Questa classe rappresenta un elemento dell'insieme \textbf{P}. Ha come variabili d'istanza \textit{u} di tipo \textit{Vertex}, che rappresenta un nodo del GSS, un intero \textit{k} e un nodo del sppf (linee 7-9). Possiede un costruttore  per inizializzare le varibili d'istanza al momento della creazione dell'oggetto (linee 11-15), dei metodi d'accesso alle variabili d'istanza (linee 17-27) ed il metodo \textit{toString()} che serve per descrivere lo stato dell'oggetto (linee 35-37). Notiamo da questa classe che è presente un nodo del sppf. Questo perchè ogni volta che viene fatto un \textit{pop()} il parser deve riprendere la costruzione dal nodo padre, che rappresenta il corpo di una produzione, a cui è stato aggiunto un nodo figlio, che sono i terminali che sono utilizzati per sostituire i non-terminali di un corpo di una produzione.\\
Ora presentiamo la classe \textbf{\textit{DescrittoreR.java}}.
\lstinputlisting{C:/Users/fabio/Documents/GitHub/GLLParsing/src/gllparsing/DescrittoreR.java}
Questa classe rappresenta un descrittore che viene memorizzato nell'insieme \textbf{R}. Ha come varibili d'istanza un'\textit{etichetta}, di tipo \textit{String}, che inidica l'item che deve essere computato, un nodo \textit{u}, che indica il nodo del GSS sul quale sta avvenendo la computazione, un intero \textit{i}, che indica un simbolo della stringa in ingresso che si vuole trovare ed un nodo \textit{w}, che indica un nodo del sppf su cui il parser deve effettuare la sostituzione con un corpo di una produzione (linee 7-10). Possiede un costruttore per inizializzare le variabili al momento della creazioe dell'oggetto (linee 12-17), dei metodi d'accesso alle variabili e il metodo \textit{toString()} per descrivere lo stato di un oggetto
\section{La classe GLL Parser}
Inn questo paragrafo descriviamo la classe \textbf{\textit{GLLParsing.java}} che definisce l'implementazione del GLL Parsing. Illustremo in più parti le varie operazioni che svolge. Il parsing viene eseguito sulla grammatica \ref{gram3} 
\begin{lstlisting}
package gllparsing;

import graph.*;
import java.io.*;
import java.util.*;
	
/*
	* GRAMMATICA
	* S->ASd
	* S->BS
	* S->epsilon
	* A->a
	* A->c
	* B->a
	* B->b
*/
public class GLLParsing {
	//gss
	private static Graph<String> gss;
	//insieme r e u che sono gli insiemi usati per registrare le scelte del non determinismo
	private static ArrayList<ElementoU> u;
	private static ArrayList<DescrittoreR>r;
	//insieme p
	private static ArrayList<ElementoP>p;
	//sppf
	private static Graph<IdNodoSppf> sppf;
	
	public static void main(String []args) {
		File f=new File("file.txt");
		Scanner buffer;
		if(f.exists()){
			try{
				FileReader in=new FileReader(f);
				buffer=new Scanner(in);
				char[] buf;
				while(buffer.hasNextLine()){
					String line=buffer.nextLine();
					buf=line.toCharArray();
					gss=new Graph<String>();
					sppf=new Graph<IdNodoSppf>();
					r=new ArrayList<DescrittoreR>();
					u=new ArrayList<ElementoU>();
					p=new ArrayList<ElementoP>();
					String esito=parse(buf);
					System.out.println(esito);
				}	
			}
			catch(Exception e){
				e.printStackTrace();
			}
		}
		else{
			System.out.println("File not Found");
		} 
	}	
\end{lstlisting}
Iniziamo a commentare le parti sostanziali. Alle linee 19-26 vengono dichiarate le variabili d'istanza che sono: il GSS ed SPPF, di tipo \textit{Graph}, che rappresentano il gss e l'sppf che viene costruito dal parser, l'insieme \textit{r} un \textit{ArrayList} che contiene gli elementi di tipo \textit{DescrittoreR}; l'insieme \textit{u}, un \textit{ArrayList} che ha elementi di tipo \textit{ElementoU} e l'insieme \textit{p}, un \textit{ArrayList} che possiede gli elementi di tipo \textit{ElementoP}. Poi viene definito il metodo \textit{main()} (linee 28-55) che avvia il parser. L'input viene inserito in un file e ogni riga di questo file, che contiene una stringa su cui fare il parsing, viene inserita in array \textit{buf}, di tipo \textit{char}, vengono istanziate le variabili d'istanza e viene chiamato il metodo \textit{parse}(), dove passiamo \textit{buf}, per fare parsing sulla stringa. Alla fine viene stampato l'esito del parsing.\\
Ora analizziamo le funzioni fondamentali.
\begin{lstlisting}
	private static void add(String etichetta, Vertex<String> nu,int j,Vertex<IdNodoSppf>cn){
		if((u.size()==0)&&(r.size()==0)){
			u.add(new ElementoU(etichetta,nu));
			r.add(new DescrittoreR(etichetta,nu,j,cn));
		}
		else{
			ElementoU el=u.get(j);
			if(!((el.getEtichetta().equals(etichetta))&&(el.getU().element().equals(nu.element())))){
				u.add(j,new ElementoU(etichetta,nu));
				r.add(new DescrittoreR(etichetta,nu,j,cn));
			}
		}
	}
\end{lstlisting}
Il metodo \textit{add}() ha come parametri un etichetta, che inidca un item che deve essere computato, un intero \textit{j}, che indica la posizione del simbolo della stringa che si vuole trovare, un nodo \textit{nu} che inidca un nodo del GSS, ed un nodo cn, che inidica un nodo dell'sppf. Il metodo aggiunge gli elementi agli insiemi \textbf{R} ed \textbf{U} se questi elementi non appartengono ad \textbf{U}[\textit{j}].
\begin{lstlisting}
	private static Vertex<String> create(String etichetta,Vertex<String> u,int j,Vertex<IdNodoSppf>cn){
		//creazione del nodo
		String nomeNodo="Ls"+j+etichetta;
		Vertex<String> v=null;
		Iterator<Vertex<String>> iteratorNodes=gss.vertices();
		while(iteratorNodes.hasNext()){
			Vertex<String> last=iteratorNodes.next();
			if(nomeNodo.equals(last.element())){
				v=last;
			}
		}
		if(v==null){
			v=gss.insertVertex(nomeNodo);
		}
		//controlliamo arco tra v ed u
		Iterator<Edge<String>> eset=gss.edges();
		boolean flag=true;
		while(eset.hasNext()){
			Edge<String> ed=eset.next();
			Vertex<String> u1=ed.getStartVertex();
			Vertex<String> u2=ed.getEndVertex();
			if((u1.element().equals(v.element()))&&(u2.element().equals(u.element()))){
				flag=false;
			}
		}
		if(flag){
			gss.insertDirectedEdge(v, u, "");
			for(ElementoP elp:p){
				if(elp.getU().element().equals(v.element())){
					add(etichetta,u,elp.getK(),elp.getZ());
				}
			}
		}
		return v;
	}
\end{lstlisting}
Il metodo create prende come parametri un \textit{etichetta}, di tipo \textit{String}, un nodo \textit{u}, che rappresenta un nodo del GSS, un intero \textit{j} che indica la posizione del simbolo della stringa in input che si vuole trovare ed un nodo \textit{cn} che rappresenta un nodo dell'sppf. Alla linea 3 definiamo il nome del nuovo nodo del gss. La rappresentazione del nodo del \textit{L}$^{j}$ descritta al paragrafo \ref{par1} avviene in questo modo: \textit{Ls}\textit{j}\textit{etichetta}. Si (linee 4-14) controlla se esiste un nodo \textit{v} nel GSS \textbf{\textit{Ls}\textit{j}\textit{etichetta}}. Se non esiste lo aggiungiamo al GSS. Alle linee 16-34, si controlla se esiste un arco tra \textit{u} e \textit{v} nel GSS. Se non esiste lo aggiungiamo e di conseguenza per ogni nodo \textit{v} presente in \textbf{P} chiama la funzione add(), passandogli \textit{etichetta}, il nodo \textit{u}, l'intero \textit{k} di un elemento di \textbf{P} il nodo del sppf \textit{z} dell'elemento \textbf{P}.
\begin{lstlisting}
	private static void pop(Vertex<String> u,int j,Vertex<String> u0,Vertex<IdNodoSppf>cn){
		//if u diverso da u0
		if(!(u.element().equals(u0.element()))){
			//mettiamo elemento u,j a p
			p.add(new ElementoP(u,j,cn));
			Iterator<Edge<String>> eset=gss.edges();
			//per ogni figlio v di aggiungi lu,v,j ad r e u
			while(eset.hasNext()){
				Edge<String> ed=eset.next();
				Vertex<String>u1=ed.getStartVertex();
				Vertex<String> v1=ed.getEndVertex();
				if(u.element().equals(u1.element())){
					add(u1.element().substring(3),v1,j,cn);
				}
			}
		}
	}
\end{lstlisting}
Il metodo \textit{pop}() ha come parametri un nodo del GSS \textit{u}, un intero \textit{j}, che indica una posizione del simbolo della stringa in input da trovare, il nodo \textit{u0}, che rapprsenta il nodo del GSS $\$$ ed il nodo dell'sppf \textit{cn}. Il metodo controlla se il nodo \textit{u} è diverso da \textit{u0}. Se lo è aggiunge \textit{u}, \textit{j} e \textit{cn} nell'insieme \textbf{P}. Poi per ogni figlio \textit{v1} di \textit{u} nel GSS chiamiamo il metodo \textit{add})() e gli passiamo come parametri l'etichetta (che sarebbe una parte del nome del nodo \textit{u}), il nodo \textit{v} del GSS, l'intero j e il nodo \textit{cn}  del sppf.
\begin{lstlisting}
	//controlla il simbolo buffer corrente di un non terminale 
	private static boolean test(char x,String nonTerm,String handle){
		if((first(x,handle))||(first('$',handle)&&(follow(x,nonTerm)))){
			return true;
		}
		else{
			return false;
		}
	}
	
	private static Vertex<IdNodoSppf> getNodeT(String nomeNodo,Vertex<IdNodoSppf>cn){
		Vertex<IdNodoSppf> v=sppf.insertVertex(new IdNodoSppf(nomeNodo));
		v.element().setId(v.hashCode());
		sppf.insertDirectedEdge(cn, v, null);
		return v;
	}

	private static Vertex<IdNodoSppf> getNodeP(int i){
		Iterator<Edge<IdNodoSppf>>it=sppf.edges();
		while(it.hasNext()) {
			Edge<IdNodoSppf>e=it.next();
			Vertex<IdNodoSppf> v1=e.getStartVertex();
			Vertex<IdNodoSppf> v2=e.getEndVertex();
			if(v2.element().getId()==i) {
				return v1;
			}
		}
		return null;
	}
\end{lstlisting}
Infine abbiamo i metodi \textit{test}(), \textit{getNodeT}(), \textit{getNodeP}(). Il metodo \textit{test}() prende come parametri \textit{x}, un simbolo della stringa in input, \textit{A}, un non-terminale di tipo \textit{String} ed \textit{item}, di tipo \textit{String}. Fa le stesse operazioni accennate al paragrafo \ref{par}. Il metodo \textit{getNodeT}() prende come parametro \textit{nomeNodo} che inidica il nome del nodo che deve avere il nuovo nodo del sppf e un nodo \textit{cn}. Questo metodo inserisce l'arco tra il nodo \textit{cn} e il nodo \textit{v} di nome \textit{nomeNodo}. Per identificare univocamente i nodi dell'sppf settiamo un id con l'\textit{hashcode} dell'oggetto. Il metodo \textit{getNodeP}() prende come parametro un intero \textit{i}, un identificativo del nodo dell'sppf. Questo metodo restituisce il nodo padre \textit{v1} che ha per figlio il nodo con l'identificativo \textit{i}.\\
Ora analizziamo il metodo \textit{parse}();
\lstinputlisting{C:/Users/fabio/Documents/GitHub/Prova.java}
Il metodo \textit{parse}() prende come parametro un array \textit{buf} che rappresenta la stringa in input su cui fare il parsing. Alle linee 3-13 vengono inizializzate le variabili usate dal parsing durante la computazione. Si inizializza l'indice \textit{i} a 0 per posizionarlo sul primo elemento della stringa, inseriamo un arco tra i nodi $\$$ e \textit{Ls0L0} nel GSS ed inseriamo il simbolo iniziale nell'sppf ed inizializza l'etichetta al simbolo iniziale. I \textbf{goto} presentati nell'algoritmo del paragrafo \ref{algoritmoIntero} sono implementati con uno \textbf{\textit{switch }}all'interno di ciclo \textbf{\textit{while}} infinito. I \textbf{\textit{case}} dello \textbf{\textit{switch}} assumono i possibili valori che può avere un etichetta. Un etichetta rappresenta un item o un non-terminale da sostituire. Le linee 18-29 abbiamo la gestione di un non-terminale non LL(1), infatti che abbiamo che se il metodo \textit{test}() ha successo, viene chiamata il metodo \textit{add}(), e ciò di applicare il non-determinismo usando l'insieme \textbf{R} ed \textbf{U}. Le linee 31-47 mostrano come viene trattato un non-terminale è LL(1). Infatti abbiamo che qui non viene applicato il non determinismo e di conseguenza questo ci permette direttamente di sostituire il non-terminale settando direttamente l'etichetta all'item opportuno. Ora analizziamo le linee 48-83. In queste linee abbiamo la gestione degli item per un corpo di una produzione. Alle linee 49-58 abbiamo che se il metodo \textit{test}() ha successo facciamo varie operazioni. La prima operazione che facciamo è creare un nodo nel GSS con l'etichetta dell'item successivo (metodo \textit{create}()), in questo modo quando verrà sostituito il simbolo puntato da questo item, sarà possibile continuare la computazione all'item successivo. Poi aggiungiamo un nodo al sppf etichettato con il corpo della produzione in questione (metodo \textit{getNodeT}()). Infine settiamo l'etichetta con il valore che ci permette di sostituire il non-terminale dell'item che si sta analizzando. Se il metodo \textit{test}() fallisce torniamo al caso base \textit{L0}. Le linee 60-68 fanno le stesse operazioni però in questo caso non chiamano il metodo \textit{getNodeT}() in quanto viene chiamato solo dal primo item del corpo della produzione. Nelle linee 70-78 si verifica se è stato trovato un simbolo della stringa in input. Se è vero incrementiamo l'indice \textit{i} in maniera tale da farlo passare al simbolo successivo. Poi viene settata l'etichetta all'item successivo. Se la verifica fallisce viene settata l'etichetta al caso base \textit{L0}. Alle linee 80-83 abbiamo completato un item, ossia il punto è alla fine del corpo della produzione, e di conseguenza viene chiamato il metodo \textit{pop}() e l'etichetta viene settata l'etichetta al caso base \textit{L0}. Alle linee 111-116 viene gestita una produzione che come corpo della produzione $\epsilon$. Viene chiamato il metodo \textit{getNodeT()} per aggiungere un nodo all'sppf, poi viene chiamato il metodo \textit{getNodeP}() per recuperare il nodo padre del nodo aggiunto precedentemente, chiamiamo il metodo \textit{pop}() e settiamo l'etichetta al caso base \textit{L0}. Osserviamo che il metodo \textit{getNodeT}() va inserito solo quando gestiamo il primo item di un corpo di una produzione e il metodo \textit{getNodeP}() va inserito solo nell'ultimo item il cui corpo di produzione è composto solo da un terminale. Ciò viene dimostrato nelle linee 152-167. Nel caso di produzione con $\epsilon$ vanno aggiunti entrambi, prima \textit{getNodeT}() e poi \textit{getNodeP}(). Alle linee 185-207 è presente la gestione del caso base \textit{L0}. Si verifica se l'insieme \textbf{R} è pieno. Se lo è viene estratta il primo descrittore da \textbf{R} e vengono impostate le variabili \textit{i}, \textit{etichetta}, il nodo del GSS e il nodo dell'sppf per poterlo computare. Dopodichè viene rimosso da \textbf{R}. Se \textbf{R} non è vuoto si verifica se l'insieme \textbf{U} è vuoto. Se lo è il parsing non ha avuto successo e termina. Altrimenti se l'ultimo descrittore di \textbf{U} ha come etichetta \textit{L0} e il nodo del GSS è $\$$ il parsing termina con successo altrimenti termina con un non successo.