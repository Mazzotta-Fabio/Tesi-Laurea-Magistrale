\chapter{GLL Parsing}
\section{Introduzione}
Nel capitolo precedente abbiamo discusso i concetti e il funzionamento del parsing su grammatiche LL(1). In questo capitolo discuteremo di un estenzione di questo parsing, chiamato \textbf{Parsing LL Generalizzato (GLL)}. Questo parsing è un parser a discesa ricorsiva ed è adatto a gestire tutte  le grammatiche ambigue e ricorsive a sinistra. In questo capitolo vedremo come questo parser supera i limiti che hanno i parser LL(1) e mostreremo i concetti base di questo parsing e del suo funzionamento.
\section{Stack e descrittori elementari}\label{par1}
In questo paragrafo discuteremo del funzionamento base del GLL Parsing. Data la seguente grammatica:
\begin{align}\label{gram3}
	S & \to ASd \mid BS \mid \epsilon \notag \\
	A & \to a \mid c \notag \\
	B & \to a \mid b 
\end{align}
Un parser a discesa ricorsiva \cite{pubblicazione: scott} è composto dalle seguenti funzioni: $p_S$(), $p_A$(), $p_B$(), la funzione principale \textit{main}() e la funzione per segnalare gli errori \textit{error}(). Ogni funzione contiene codice per ogni alternativa, $\alpha$, e verificano il simbolo corrente della stringa in input appartiene a FIRST($\alpha$) o al FOLLOW($\alpha$). La stringa in input viene rappresentata come un array globale \textit{I} di lunghezza \textit{m}+1, dove \textit{I}[\textit{m}]=$\$$, segnala la fine della stringa. L'implementazione del parser viene rappresentata di seguito.\\
main()$\{$ \textit{i} = 0 \par
\hspace{1cm}\textbf{if}(\textit{I}[\textit{i}] $\in$ $\{$\textit{a, b, c, d,$\$$}$\}$)$\{$ $p_S$(); \textbf{else} \textit{error}();\par
\hspace{1cm}\textbf{if}(\textit{I}[\textit{i}] = $\$$)$\{$ report success $\}$ \textbf{else} \textit{error}()\\	
$\}$\\
$p_S$()$\{$ \par
\hspace{0.5cm}\textbf{if}(\textit{I}[\textit{i}] $\in$ $\{$\textit{a, c}$\}$)$\{$ $p_A$(); $p_S$(); \textbf{if}(\textit{I}[\textit{i}] = \textit{d})$\{$ \textit{i} = \textit{i} + 1; $\}$ \textbf{else} \textit{error}(); $\}$\par
\hspace{0.5cm}\textbf{if}(\textit{I}[\textit{i}] $\in$ $\{$a, b$\}$)$\{$ $p_B$(); $p_S$(); $\}$ $\}$\\
$p_A$()$\{$ \par
\hspace{0.5cm}\textbf{if}(\textit{I}[\textit{i}] = \textit{a})$\{$ \textit{i} = \textit{i} + 1;$\}$\par \hspace{0.5cm}\textbf{else} \textbf{if}(\textit{I}[\textit{i}] = \textit{c})$\{$ \textit{i} = \textit{i} + 1 $\}$ \textbf{else} \textit{error}(); $\}$\\
$p_B$()$\{$ \par \hspace{0.5cm}\textbf{if}(\textit{I}[\textit{i}] = \textit{a})$\{$ \textit{i} = \textit{i} + 1; $\}$\par \hspace{0.5cm}\textbf{else} \textbf{if}(\textit{I}[\textit{i}] = \textit{b})$\{$ \textit{i} = \textit{i} + 1 $\}$ \textbf{else} \textit{error}(); $\}$\par
\vspace{0.3cm}
Questa è la tabella di parsing della grammatica \ref{gram3}.
\begin{table}[hbpb]
	\centering
	\label{tabellaparsingNLL1}
	\begin{tabular}{cccccc} 
		\toprule
		%\multirow{2}*{\textbf{Non Terminale}} & %\multicolumn{4}{c}{\textbf{Simbolo d'ingresso}} \\ 
		%\cmidrule(lr){2-4}
		& a & b & c & d & $\$$ \\ 
		\midrule
		\textit{S} 	& \textit{S}$\to$\textit{ASd} $\mid$ \textit{BS} & \textit{S}$\to$\textit{BS}&\textit{S}$\to$ \textit{ASd}&\textit{S}$\to$$\epsilon$& \textit{S}$\to$$\epsilon$\\ 
		\textit{A} & \textit{A}$\to$\textit{a}&  & \textit{A}$\to$\textit{c}\\ 
		\textit{B} & \textit{B}$\to$\textit{a} & \textit{B}$\to$\textit{b}&   \\ 
		\bottomrule
	\end{tabular}
	\caption{\textit{Tabella di parsing della grammatica }\ref{gram3}}
\end{table} \par
Da quello che si può notare dalla tabella \ref{gram3} questa grammatica non è LL(1) in quanto è presente un conflitto e di conseguenza l'algoritmo implementato non funziona correttamente. Affinchè l'algoritmo funzioni correttamente è necessario aggiungere il non-determinismo. Per fare ciò dobbiamo convertire le chiamate a funzioni con operazioni di \textbf{push} su  uno stack e utilizzare i \textbf{goto}. Poi partizioniamo in varie parti i corpi delle funzioni il cui non-terminale non è LL(1) ed attribuiamo un etichetta ad ogni partizione. In questo caso abbiamo più partizioni per \textit{S}. Per registrare le possibili scelte che il parser può fare per sostituire un non-terminale utilizziamo dei \textbf{descrittori} all'interno dell'algoritmo a discesa ricorsiva e sostituiamo il punto in cui termina l'algoritmo con l'esecuzione di un descrittore successivo. Le funzioni d'errore vengono sostituite con l'esecuzione di descrittori successivi. Il nuovo punto di termine sarà quando non esistono più descrittori da eseguire. Formalmente un \textbf{descrittore elementare} è una tripla (\textbf{L,s,j}) dove \textbf{L} è un etichetta, \textbf{s} è uno stack e \textbf{j} è la posizione nell'array \textit{I}. Quessti descrittori li manteniamo in un insieme \textbf{R}. Ogni volta che si verifica la fine di una funzione di parsing e ad ogni punto in cui è presente un terminale non LL(1) (quindi siamo in presenza di non-determinismo) all'interno dell'algoritmo, creiamo un nuovo descrittore che è formato dall'etichetta in cima allo stack corrente. Quando l'algoritmo di parsing trova un simbolo del'input \textit{I}[\textit{i}] diciamo che l'etichetta \textit{L} in cima allo stack è estratto dallo stack \textit{s}=[\textit{s$^{'}$},\textit{L}] e (\textit{L,s$^{'}$,i}) viene aggiunta a \textbf{R}. Questa azione viene denotata con la funzione \textit{pop(s,i,\textbf{R})}. Dopo aver fatto ciò rimuoviamo il descrittore (\textit{L$^{'}$, t, j}) da \textbf{R} e l'algoritmo riparte dall'etichetta \textit{L$^{'}$}, con stack \textit{t} e con il simbolo in input \textit{I}[\textit{j}]. L'algoritmo termina quando l'insieme \textbf{R} è vuoto. Useremo la notazione \textit{L}$^{k}$ per unire l'etichetta \textit{L} e l'indice \textit{k} che indica il simbolo corrente nell'input \textit{I}; mentre lo stack vuoto viene denotato con [ ]. Lo stack \textit{s} viene aggiornato con la funzione \textit{push(s, L$^{k}$)}; questa funzione non fa altro che aggiungere  l'elemento \textit{L}$^{k}$ in cima allo stack. Di seguito viene presentato l'algoritmo.\par
\vspace{0.5cm}
\hspace{0.2cm}i = 0; \textbf{R} = $\emptyset$; s = [L$_0^{0}$];\\
$L_S$: \textbf{if}(\textit{I}[\textit{i}] $\in$ $\{$\textit{a, c}$\}$) add($L_{S1}$, s, i) to \textbf{R} \par
\hspace{0.4cm}\textbf{if}(\textit{I}[\textit{i}] $\in$ $\{$\textit{a, b}$\}$) add($L_{S2}$, s, i) to \textbf{R}\par
\hspace{0.4cm}\textbf{if}(\textit{I}[\textit{i}] $\in$ $\{$\textit{d, $\$$}$\}$) add($L_{S3}$, s, i) to \textbf{R} \\
$L_0$: \textbf{if}(\textbf{R} $\ne$ $\emptyset$)$\{$ remove(L, $s_1$, j) from \textbf{R} \par
\hspace{0.4cm}\textbf{if}(L = $L_0$ and $s_1$ = [ ] and j = |I|) report success \par
\hspace{0.4cm}\textbf{else}$\{$ s = $s_1$; i = j; \textbf{goto} L $\}$ \\
$L_{S1}$: \textit{push}(s, L$_1^{i}$); \textbf{goto} $L_A$\\
$L_1$:  \textit{push}(s, L$_2^{i}$); \textbf{goto} $L_S$\\
$L_2$:  \textbf{if}(\textit{I}[\textit{i}] = a)$\{$ i = i + 1; \textit{pop}(s, i, \textbf{R});$\}$ \textbf{goto} $L_0$ \\
$L_{S2}$: \textit{push}(s, L$_3^{i}$); \textbf{goto} $L_B$\\
$L_3$:  \textit{push}(s, L$_4^{i}$); \textbf{goto} $L_S$\\
$L_4$: \textit{pop}(s, i, \textbf{R}); \textbf{goto} $L_0$ \\
$L_{S3}$: \textit{pop}(s, i, \textbf{R}); \textbf{goto} $L_0$ \\
$L_A$:  \textbf{if}(\textit{I}[\textit{i}] = a)$\{$ i = i + 1; \textit{pop}(s, i, \textbf{R});$\}$ \textbf{goto} $L_0$ $\}$ \par
\hspace{0.2cm} \textbf{else}$\{$ \textbf{if}(\textit{I}[\textit{i}] = c)$\{$ i = i + 1; \textit{pop}(s, i, \textbf{R});$\}$ \par 
\hspace{1.5cm}\textbf{goto} $L_0$ $\}$ \\
$L_B$:  \textbf{if}(\textit{I}[\textit{i}] = a)$\{$ i = i + 1; \textit{pop}(s, i, \textbf{R});$\}$ \textbf{goto} $L_0$ $\}$ \par
\hspace{0.2cm} \textbf{else}$\{$ \textbf{if}(\textit{I}[\textit{i}] = b)$\{$ i = i + 1; \textit{pop}(s, i, \textbf{R});$\}$ \par
\hspace{1.5cm}\textbf{goto} $L_0$ $\}$ \par 
\vspace{0.3cm}
Ora facciamo un esempio ed eseguiamo l'algoritmo con la stringa d'input \textit{aad}$\$$. Incomiciamo la nostra computazione aggiungendo prima la tripla (L$_{S1}$,[L$_0^{0}$],0), e poi la tripla (L$_{S2}$,[L$_0^{0}$],0) ad \textbf{R} ed andiamo all'etichetta L$_0$. Rimuoviamo (L$_{S1}$,[L$_0^{0}$],0) da \textbf{R} ed andiamo alla linea con etichetta L$_{S1}$. L'operazione di \textit{push} aggiunge allo stack \textit{s} [L$_0^{0}$,L$_1^{0}$] ed andiamo in L$_A$. In L$_A$ abbiamo che il parser ha trovato una coincidenza con il simbolo \textit{a}, incrementa l'indice \textit{i}, usato per leggere i simboli dell'input, e fa un operazione di \textit{pop} ed aggiunge  (L$_{1}$,[L$_0^{0}$],1) ad \textbf{R} e ritorna in L$_0$. Allo stesso modo processiamo  (L$_{S2}$,[L$_0^{0}$],0) da \textbf{R} e alla fine avremo (L$_{3}$,[L$_0^{0}$],1) che viene aggiunto ad \textbf{R}. Quindi \textbf{R} risulterà avere le seguenti triple:\par
\hspace{1.5cm} \textbf{R} = $\{$ (L$_{1}$,[L$_0^{0}$],1), (L$_{3}$,[L$_0^{0}$],1) $\}$\par
\vspace{0.3cm}Successivamente estraiamo (L$_{1}$,[L$_0^{0}$],1) e lo processiamo. Alla linea L$_1$ facciamo un \textit{push} sullo stack s [L$_0^{0}$,L$_2^{1}$] ed andiamo alla linea con etichetta L$_S$ e aggiungiamo (L$_{S1}$,[L$_0^{0}$,L$_2^{1}$],1) e (L$_{S2}$,[L$_0^{0}$,L$_2^{1}$],1) ad \textbf{R}. Allo stesso modo processiamo (L$_{3}$,[L$_0^{0}$],1) e in \textbf{R} abbiamo:\par
\vspace{0.3cm}\textbf{R} = $\{$ (L$_{S1}$,[L$_0^{0}$,L$_2^{1}$],1), (L$_{S2}$,[L$_0^{0}$,L$_2^{1}$],1), (L$_{S1}$,[L$_0^{0}$,L$_4^{1}$],1), (L$_{S2}$,[L$_0^{0}$,L$_4^{1}$],1) $\}$\par
\vspace{0.3cm} Processando ognuno di questi elementi otteniamo:\par 
\vspace{0.3cm} \textbf{R} = $\{$  (L$_{1}$,[L$_0^{0}$,L$_2^{1}$],2), (L$_{3}$,[L$_0^{0}$,L$_2^{1}$],2),  (L$_{1}$,[L$_0^{0}$,L$_4^{1}$],2) (L$_{3}$,[L$_0^{0}$,L$_4^{1}$],2) $\}$\par 
\vspace{0.3cm}Quando l'input risulta essere \textit{I}[\textit{i}] = \textit{d} e processiamo queste triple otteniamo che in \textbf{R} abbiamo:\par 
\vspace{0.3cm}\textbf{R} = $\{$ (L$_{S3}$,[L$_0^{0}$,L$_2^{1}$,L$_2^{2}$],2), (L$_{S3}$,[L$_0^{0}$,L$_2^{1}$,L$_4^{2}$],2),  (L$_{S3}$,[L$_0^{0}$,L$_4^{1}$,L$_4^{2}$],2) (L$_{S3}$,[L$_0^{0}$,L$_4^{1}$,L$_4^{2}$],2)$\}$\par 
\vspace{0.3cm} Processando di nuovo questi elementi otteniamo:\par 
\vspace{0.3cm}\hspace{0.5cm}\textbf{R} = $\{$ (L$_{2}$,[L$_0^{0}$,L$_2^{1}$],2), (L$_{4}$,[L$_0^{0}$,L$_2^{1}$],2),  (L$_{2}$,[L$_0^{0}$,L$_4^{1}$],2) (L$_{4}$,[L$_0^{0}$,L$_4^{1}$],2)$\}$\par 
\vspace{0.3cm} Da queste triple otteniamo:\par 
\vspace{0.3cm}\hspace{0.5cm}\textbf{R} = $\{$ (L$_{2}$,[L$_0^{0}$],3), (L$_{2}$,[L$_0^{0}$],2),  (L$_{4}$,[L$_0^{0}$],3) (L$_{4}$,[L$_0^{0}$],2)$\}$\par 
\vspace{0.3cm} Quando abbiamo \textit{I}[3] = $\$$ processiamo le triple mostrate precedentemente ed otteniamo la tripla (L$_{0}$,[ ],3) e la tripla (L$_{0}$,[ ],2) che vengono aggiunte ad \textbf{R} e di conseguenza l'algoritmo termina con un successo.
\section{Graph structured stacks}
Nel paragrafo precedente abbiamo visto che ogni volta si trovava un non-terminale non LL(1) applicavamo il non-determinismo duplicando lo stack usato dal parser. Per rappresentare tutti questi stack in unica struttura dati utilizzeremo il \textbf{Graph structured stacks}(\textbf{GSS}). Viene definito \cite{tesi: lr} come un grafo diretto aciclico (DAG) i cui nodi hanno etichette L$^{k}$ che corrispondono agli elementi usati dallo stack, dove L indica un etichetta di una linea di codice e \textit{k} la posizione su un simbolo della stringa in input. Questi nodi sono raggruppati in vari insiemi disgunti chiamati \textit{livelli}. Il GSS viene costruito un livello alla volta. Infatti ogni qualvolta il parser trova una coincidenza tra un simbolo in input e la grammatica crea un livello. Il GSS viene disegnato da sinistra verso destra ed il nodo più a sinitra rappresenta la cima di ogni stack. Per costruire il GSS all'interno dell'algoritmo di parsing, dobbiamo usare una nuova tripla, chiamato \textbf{descrittore}. Un descrittore è formata da (\textit{L}, \textit{u}, \textit{i}), dove \textit{L} è un etichetta, \textit{u} è un nodo del GSS e \textit{i} che indica il simbolo della stringa in input \textit{I}[ ]. Queste triple vengono aggiunte ad \textbf{R}.
\subsection{Insiemi U e P}
Per costruire il GSS è necessario utilizzare un altro insieme \textbf{U} che contiene gli stessi descrittori che inseriamo nell'insieme \textbf{R}. Infatti abbiamo che U$_i$ = $\{$ (\textit{L}, \textit{u}) $\mid$ (\textit{L}, \textit{u}, \textit{i}) è stato aggiunto ad \textbf{R}$\}$. Un problema \cite{pubblicazione: scott} che può sorgere sia ha quando un nodo figlio \textit{w} è aggiunto ad \textit{u} dopo aver eseguito le operazioni di pop sul GSS perchè l'azione di pop necessita di essere applicata a questo nodo figlio. Per risolvere ciò usiamo l'insieme \textbf{P} che contiene coppie (\textit{u}, \textit{k}) e verranno utilizzate per eseguire le operazioni di pop. Infatti quando un nuovo nodo figlio \textit{w} è aggiunto ad \textit{u}, ogni elemento (\textit{u}, \textit{k}) presente in \textbf{P}, se (L$_u$, \textit{w}) non è presente in \textbf{U$_k$}, allora (L$_u$, \textit{u}, \textit{k}) è aggiunto ad \textbf{R}, dove L$_u$ è l'etichetta nel nodo \textit{u}. L'implementazione di questa tecnica verrà mostrata nel paragrafo \ref{par}
\section{Definizione GLL Parsing}
In questo paragrafo vengono definite le basi per realizzare l'algoritmo di GLL Parsing. Mostreremo come costruire il GSS e il suo l'output. Il principio base è quello mostrato al paragrafo \ref{par1}
\subsection{Funzioni Fondamentali}\label{par}
L'algoritmo \cite{pubblicazione: scott} utilizza quattro funzioni che sono essenziali per il suo funzionamento. Queste funzioni vengono elencate qui di seguito.
\begin{itemize}
	\item \textbf{\textit{add()}}: La funzione \textit{add}(\textit{L}, \textit{u}, \textit{j}) controlla se c'è un descrittore (\textit{L}, \textit{u}) in U$_j$ e se non c'è lo aggiunge a U$_j$ e ad \textbf{R}. La funzione è definita nel seguente modo:\par 
	\textit{add}(\textit{L}, \textit{u}, \textit{j})$\{$ \textbf{if}((\textit{L}, \textit{u}) $\notin$ U$_j$)$\{$ add (\textit{L}, \textit{u}) to U$_j$; add (\textit{L}, \textit{u}, \textit{j}) to \textbf{R} $\}$ $\}$
	\item \textbf{\textit{pop()}}: La funzione \textit{pop}(\textit{u}, \textit{j}) chiama la funzione \textit{add}(\textit{L$_u$}, \textit{v}, \textit{j}) per tutti i figli \textit{v} di \textit{u} e aggiunge (\textit{u}, \textit{j}) a \textbf{P}. Viene definita nel seguente modo: \par
	\textit{pop}(\textit{u}, \textit{j})$\{$ \textbf{if}(\textit{u} $\ne$ \textit{u$_0$})$\{$ add (\textit{u}, \textit{j}); \par \hspace{4.1cm}\textbf{for} each child \textit{v} of \textit{u} $\{$ add (\textit{L$_u$}, \textit{v}, \textit{j}); $\}$ $\}$ $\}$
	\item \textbf{\textit{create()}}: La funzione \textit{create}(\textit{L}, \textit{u}, \textit{j}) crea un nodo \textit{v} nel GSS etichettato \textit{L}$^{j}$ con figlio \textit{u} se non esiste ancora e restituisce \textit{v}. Se (\textit{v}, \textit{k}) appartiene a \textbf{P} chiama la funzione \textit{add}(\textit{L}, \textit{u}, \textit{k}). La definizione di questa funzione è al seguente:\par 
	\textit{create}(\textit{L}, \textit{u}, \textit{j})$\{$ \textbf{if} there is not a GSS node labelled \textit{L$^{j}$} create one\par
	\hspace{3cm} let \textit{v} be the GSS node labelled \textit{L$^{j}$} \par
	\hspace{3cm} \textbf{if} there is not an edge from \textit{v} to \textit{v}$\{$ \par
	\hspace{3.6cm} create an edge from \textit{v} to \textit{u}  \par
	\hspace{3.6cm} \textbf{for all} ((\textit{v, k}) $\in$ \textbf{\textbf{P}}) $\{$ \textit{add}(\textit{L}, \textit{u}, \textit{k}) $\}$ $\}$ \par  
	\hspace{3cm} \textbf{return} \textit{v} $\}$
	\item \textbf{\textit{test()}}: La funzione \textit{test}(\textit{x}, \textit{A}, $\alpha$) controlla se il simbolo d'input corrente \textit{x} appartiene al FOLLOW(\textit{A}), dove \textit{A} è un non-terminale o appartiene al FIRST($\alpha$), dove $\alpha$ è un item che stiamo processando. \`E definita nel seguente modo:\par
	\textit{test}(\textit{x},\textit{A},$\alpha$)$\{$ \textbf{if}(\textit{x} $\in$ FIRST($\alpha$)) or ($\epsilon$ $\in$ FIRST($\alpha$) and \textit{x} $\in$ FOLLOW(\textit{A}))$\{$ \par 
	\hspace{2.6cm}\textbf{ return true } $\}$\par 
	 \hspace{2.1cm}\textbf{else} $\{$ \textbf{return false} $\}$ $\}$
\end{itemize}
\subsection{Gestione degli item}
Informalmente un \textbf{item} \cite{libro: compilatori} di una grammatica \textit{G} è una produzione di \textit{G} con un punto in qualche posizione del corpo. Ad esempio la produzione \textit{C} $\to$ \textit{DKL} ammette quattro item:
\begin{align}
	\textit{C} & \to \textit{.DKL} \notag \\
	\textit{C} & \to \textit{D.KL} \notag \\
	\textit{C} & \to \textit{DK.L} \notag \\
	\textit{C} & \to \textit{DKL.} \notag 
\end{align}
La produzione \textit{C} $\to$ $\epsilon$ genera l'item \textit{C}$\to$. .\\
Un item indica una porzione di una produzione che si sta analizzando ad un certo punto del processo di parsing. Sia $\alpha$ la parte di produzione dopo il punto di un item. Nel GLL Parsing gli item verrano gestiti nel seguente modo: 
\begin{enumerate}
	\item Un item del tipo \textit{a}$\alpha$, dove \textit{a} è un terminale, definiamo:\par
	\textit{code}(\textit{a}$\alpha$, j) = \textbf{if}(\textit{I}[\textit{j}] = \textit{a})$\{$ \textit{j} = \textit{j} + 1 $\}$ \textbf{else} $\{$ \textbf{goto} L$_0$ $\}$
	\item Un item del tipo \textit{A}$\alpha$, dove \textit{A} è un non-terminale, definiamo:\par
	\textit{code}(\textit{A}$\alpha$, j,\textit{X}) = \textbf{if}(\textit{test}(\textit{I}[\textit{j}],\textit{X},\textit{A}$\alpha$))$\{$ \par 
	\hspace{4cm}\textit{c}$_u$ = \textit{create}(\textit{R}$_A\alpha$,\textit{c}$_u$,\textit{j}); \textbf{goto} \textit{L}$_A$ $\}$ \par 
	\hspace{3cm}\textbf{else} $\{$ \textbf{goto} L$_0$ $\}$
	\item Per la produzione \textit{A}$\to$$\epsilon$ con item \textit{C}$\to$. definiamo: \par
	\hspace{1cm}\textit{code}(\textit{A}$\to$$\epsilon$, j) = \textit{pop}(\textit{c}$_u$,\textit{j}); \textbf{goto} L$_0$;
\end{enumerate}
\vspace{0.5cm}Quindi, in base a ciò, per ogni produzione \textit{A} $\to$ $\beta$, dove $\beta$=\textit{x}$_1$$\dots$\textit{x}$_n$, abbiamo:
\begin{enumerate}
	\item Se \textit{x}$_1$ è un terminale:\par 
	\hspace{2.5cm}\textit{code}(\textit{A}$\to$$\beta$, j) = \hspace{2cm}\textit{j} = \textit{j} + 1\par
	\hspace{7.3cm} \textit{code}(\textit{x}$_2$$\dots$\textit{x}$_n$, \textit{j},\textit{A})\par
	\hspace{7.3cm} \textit{code}(\textit{x}$_3$$\dots$\textit{x}$_n$, \textit{j},\textit{A})\par
	\hspace{7.3cm} \dots \par
	\hspace{7.3cm} \textit{code}(\textit{x}$_n$, \textit{j},\textit{A})\par 
	\hspace{7.3cm} \textit{pop}(\textit{c}$_u$,\textit{j}); \textbf{goto} L$_0$;
	\item Se \textit{x}$_1$ è un non-terminale:\par 
	\hspace{2.2cm}\textit{code}(\textit{A}$\to$$\beta$, j) = \hspace{2cm} \textit{c}$_u$=\textit{create}(\textit{R}$_{Al}$,\textit{c}$_u$,\textit{j}); \textbf{goto} \textit{L}$_A$\par
	\hspace{7.1cm} $A_l$: \textit{code}(\textit{x}$_2$$\dots$\textit{x}$_n$, \textit{j},\textit{A})\par
	\hspace{7.8cm} \textit{code}(\textit{x}$_3$$\dots$\textit{x}$_n$, \textit{j},\textit{A})\par
	\hspace{7.8cm} \dots \par
	\hspace{7.8cm} \textit{code}(\textit{x}$_n$, \textit{j},\textit{A})\par 
	\hspace{7.8cm} \textit{pop}(\textit{c}$_u$,\textit{j}); \textbf{goto} L$_0$;
\end{enumerate}
\subsection{Gestione delle produzioni}
Siano \textit{A} $\to$ $\alpha_1$ $\mid$ $\dots$ $\mid$ $\alpha_{n}$ regole grammaticali. Quando dobbiamo gestire le sostituzioni dei non-terminali con gli opportuni corpi di produzione durante il parsing vengono definiti due modi:
\begin{enumerate}
	\item Se \textit{A} è un non-terminale LL(1) che non presenta conflitti abbiamo:\par
	\hspace{2.5cm}\textit{code}(\textit{A}, j) = \hspace{2cm}\textbf{if}(\textit{test}(\textit{I}[\textit{j}],\textit{X},$\alpha_1$)$\{$ \textbf{goto} \textit{L}$_{A1}$ $\}$ \par 
	\hspace{6.8cm} \dots \par
	\hspace{6.8cm} \textbf{else if}(\textit{test}(\textit{I}[\textit{j}],\textit{X},$\alpha_n$)$\{$\textbf{goto} \textit{L}$_{An}$$\}$ \par 
	\hspace{6.2cm} \textit{L}$_{A1}$: \textit{code}(\textit{A}$\to$$\alpha_1$, j) \par 
	\hspace{6.2cm} \dots \par 
	\hspace{6.2cm} \textit{L}$_{An}$: \textit{code}(\textit{A}$\to$$\alpha_{n}$ , j) \par 
	\item Se \textit{A} è un non-terminale non LL(1) che presenta conflitti abbiamo:\par
	\hspace{2.5cm}\textit{code}(\textit{A}, j) = \hspace{2cm}\textbf{if}(\textit{test}(\textit{I}[\textit{j}],\textit{X},$\alpha_1$)$\{$ \textit{add}(\textit{L}$_{A1}$,\textit{c}$_u$,\textit{j})$\}$ \par 
	\hspace{6.8cm} \dots \par
	\hspace{6.8cm} \textbf{else if}(\textit{test}(\textit{I}[\textit{j}],\textit{X},$\alpha_n$)$\{$ \textit{add}(\textit{L}$_{An}$,\textit{c}$_u$,\textit{j})$\}$ \par 
	\hspace{6.8cm}\textbf{goto} \textit{L}$_0$ \par 
	\hspace{6.2cm} \textit{L}$_{A1}$: \textit{code}(\textit{A}$\to$$\alpha_1$, j) \par 
	\hspace{6.2cm} \dots \par 
	\hspace{6.2cm} \textit{L}$_{An}$: \textit{code}(\textit{A}$\to$$\alpha_{n}$ , j) \par 
\end{enumerate}
\subsection{Shared packed parse forests}
Gli alberi sintattici prodotti da una grammatica ambigua \cite{tesi: lr} possono essere combinati in unica struttura chiamata \textbf{Shared packed parse forests} (Sppf). I nodi padri sono uniti in nuovo nodo ed un nodo involucro diventa il nodo padre di ogni sottoalbero. Per la grammatica \ref{grammaticaEspressioni}, che può essere rappresentata con due alberi di parsing, il corrispondente sppf è mostrato nella figura 3.1. 
\begin{figure}[hbpb]
	\centering
	\begin{forest}
	[E
	[E[\textbf{id}]] [+] [E[E[\textbf{id}]] [*][E[\textbf{id}]]]
	[E[E[\textbf{id}]][+][E[\textbf{id}]]] [*] [E[\textbf{id}]]
	]
	\end{forest}
	\caption{\textit{Sppf relativo alla stringa} \textbf{id}+\textbf{id}*\textbf{id} }\label{sppf}
\end{figure}
Il parsing GLL restituirà in output un sppf in quanto opera su grammatiche ambigue e ricorsive a sinistra. Per poter costruire questo albero sono state definite due funzioni:
\begin{itemize}
	\item \textbf{\textit{getNodeT()}}: la funzione \textit{getNodeT}($\alpha$,\textit{c}$_n$) viene usata per creare un nodo, etichettato $\alpha$, all'interno del sppf e collegarlo come nodo figlio al nodo \textit{c}$_n$. Viene chiamata quando il parser trova un terminale e quando sostituiamo un non-terminale con il corpo della produzione. Viene definita nel seguente modo:\par
	\hspace{1.5cm}\textit{getNodeT}($\alpha$,\textit{c}$_u$)$\{$ create a new node \textit{c}$_v$ in SPPF; \par 
	 \hspace{4.6cm}create an adge from \textit{c}$_u$ to \textit{c}$_v$ $\}$
	 \item \textbf{\textit{getNodeP()}}: la funzione \textit{getNodeP}(\textit{c}$_u$) viene usata per ottenere un nodo \textit{c}$_v$ padre del nodo \textit{c}$_u$ all'interno del sppf. Viene chiamata quando è necessario recuperare un non-terminale ogni volta che viene trovato un simbolo della stringa in input. In questo modo recuperiamo i restanti non-terminali del corpo di una produzione. Viene definita nel seguente modo:\par
	 \hspace{1.5cm}\textit{getNodeP}(\textit{c}$_u$)$\{$ \textbf{for} each edge E(\textit{c}$_p$,\textit{c}$_c$)in Sppf $\{$  \par 
	 \hspace{4.4cm}\textbf{if}(\textit{c}$_u$=\textit{c}$_c$)$\{$ \textbf{return} \textit{c}$_c$ $\}$ $\}$ $\}$
\end{itemize}
\section{Costruzione del GLL Parser}
Ora, avendo definito le varie funzioni e le varie operazioni da eseguire, siamo in grado di costruire in GLL Parser. Qui di seguito riportiamo lo pseudocodice del parser e nei capitoli successivi ne vedremo l'implementazione in Java di questo parser.\par 
\vspace{0.2cm}
Sia $\Gamma$ una grammatica \cite{pubblicazione: scott} e i suoi non-terminali sono \textit{A}, $\dots$, \textit{X}. L'algoritmo di GLL parsing per la grammatica $\Gamma$ risulta essere il seguente:\par
\vspace{0.5cm}
\textit{m} è una costante che indica la lunghezza della stringa in input\\
\textit{I} è un array che contiene la stringa in input di dimensione \textit{m}+1\\
\textit{i} è una variabile intera usata per accedere alle locazioni dell'array \textit{I}\\
GSS è un DAG che contiene i nodi etichettati nella forma \textit{L$^{j}$}\\
\textit{c}$_u$ è un nodo del GSS\\
\textbf{P} è un insieme che ha coppie formate da un nodo del GSS e di intero\\
\textbf{R} è un insieme di descrittori \par
\vspace{0.5cm}

read the input into \textit{I} and set \textit{I}[\textit{m}]=$\$$, \textit{i}=0;\\
create GSS nodes \textit{u}$_1$=\textit{L}\textit{L}$_0^{0}$, \textit{u}$_0$=$\$$ and edge (\textit{u}$_0$,\textit{u}$_1$)\\
\textit{c}$_u$=\textit{u}$_1$, \textit{i}=0\\
\textbf{for} 0 $\le$ \textit{j} $\le$ \textit{m} $\{$ \textbf{U$_j$} = $\emptyset$ $\}$\\
\textbf{R}=$\emptyset$, \textbf{P}=$\emptyset$\\
\textbf{goto} \textit{L}$_S$;\\
\textit{L}$_0$: \textbf{if}(\textbf{R} $\ne$ $\emptyset$)$\{$\par
\hspace{2cm} remove a descriptor (\textit{L},\textit{u},\textit{j}) from \textbf{R}\par
\hspace{2cm} \textit{c}$_u$=\textit{u}, \textit{i}=\textit{j} , \textbf{goto} \textit{L} $\}$\par
\hspace{0.1cm} \textbf{else if}((\textit{L}$_0$,\textit{u}$_0$,\textit{j}) $\in$ \textit{U}$_m$)$\{$ report success $\}$ \textbf{else}$\{$ report failure $\}$\\
\textit{L}$_A$: \textit{code}(\textit{A},\textit{i})\\
\dots \\
\textit{L}$_X$: \textit{code}(\textit{X},\textit{i})