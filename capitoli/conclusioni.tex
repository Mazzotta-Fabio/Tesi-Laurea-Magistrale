\chapter{Conclusioni}
\section{Obiettivi raggiunti}
Gli obiettivi di questa prima parte di lavoro era quello di introdurre il parsing top down descrivendo il parsing LL(1) e ne è stato delineato il suo funzionamento e i suoi limiti, in quanto non riesce a gestire grammatiche ambigue e ricorsive. Per superare questi limiti abbiamo introdotto il parsing GLL che utilizza il non determinismo per gestire grammatiche ambigue e ricorsive. Il suo utilizzo permette di gestire tutti i conflitti dell'LL(1) processando tutte sostituzioni possibili per un non terminale. Questo parsing usa una struttura dati molto potente per gestire la computazione del non determinismo ed è il GSS. Il risultato prodotto da questo parsing è l'Sppf, un albero che combina tutti i risultati dei vari parser creati dalla computazione del GLL. L'obiettivo finale di questo lavoro ha portato a creare un estensione del GLL Parsing per grammatiche posizionali utilizzando sempre lo stesso algoritmo per processare i non-terminali e gli item delle varie produzioni, però ne è stata modificata la gestione della lettura dei token successivi in quanto non avveniva più in maniera lineare, dove i token vengono letti in successione, ma avviene in base alle regole definite dagli operatori spaziali che possono diversi per ogni grammatica posizionale.