\chapter{Conclusioni}
\section{Obiettivi raggiunti}
Gli obiettivi della prima parte della tesi sono stati quelli di introdurre il parsing top down, descrivendone il funzionamento del parsing LL(1) e delineandone i suoi limiti, in quanto non riesce a gestire grammatiche ambigue e ricorsive. Per superare questi limiti abbiamo introdotto il parsing GLL che utilizza i principi del non-determinismo per gestire tutte le grammatiche comprese quelle ambigue e ricorsive. Il suo funzionamento permette di gestire tutti i conflitti presenti nella tabella di parsing LL(1); ciò viene fatto creando dei nuovi flussi di computazione per ogni conflitto di sostituzione. Questo parsing usa una struttura dati, chiamata GSS, che combina i vari stack usati dai vari flussi di computazione. Il risultato prodotto da questo parsing è l'SPPF, un albero che combina in un unica struttura tutti gli alberi sintattici creati dai flussi di computazione del GLL. L'obiettivo finale di questo lavoro ha portato a creare un estensione del GLL Parsing per grammatiche posizionali. Viene utilizzato sempre lo stesso algoritmo per processare le grammatiche, però ne è stata modificata la gestione della lettura dei simboli successivi in quanto non avviene più in maniera lineare, dove i simboli vengono letti in successione da sinistra verso destra, ma avviene in base alle direzioni definite dalle relazioni spaziali che possono diverse per ogni grammatica posizionale.