\chapter{GLL Parsing Posizionale}
\section{Introduzione}
In questo capitolo introduciamo un estensione del GLL parsing, il \textbf{GLL Parsing Posizionale}. Il GLL Parsing Posizionale permette di trattare le \textit{grammatiche posizionali} che producono i cosidetti \textbf{linguaggi non lineari}. Queste grammatiche presentano produzioni contenenti \textit{relazioni spaziali}, ossia relazioni che indicano come deve essere letto l'input. Illustreremo la definizione base di questa nuova grammatica e il funzionamento base delle relazioni spaziali.
\section{Definizione formale}
Una \textbf{grammatica posizionale context-free} è sestupla \cite{pubblicazione: tomita} i cui elementi sono:
\begin{enumerate}
	\item \textbf{Non-Terminali (N)}: variabili sintattiche che denotano un insieme di stringhe.
	\item \textbf{Terminali (T)}: simboli di base che definiscono il linguaggio.
	\item \textbf{Simbolo iniziale (S)}: è un non-terminale e l'insieme di stringhe che esso denota coincide con l'intero linguaggio generato dalla grammatica.
	\item \textbf{Produzioni (P)}: regole che definiscono come possono essere combinati i terminali e i non-terminali.
	\item \textbf{Relazioni Spaziali (POS)}: relazioni che danno informazioni sulle posizioni di spostamento da effettuare per la lettura dell'input.
	\item \textbf{Regole di valutazione (PE)}: regole usate per determinare come deve essere effettuato lo spostamento sull'input.
\end{enumerate}
Ogni produzione presenta la seguente forma:
\begin{align}
	A \to \alpha_{1} REL_1 \alpha_{2} REL_2 \dots REL_{m-1} \alpha_{n} \quad m \ge 1 \notag 
\end{align}
dove \textit{A} $\in$ N, ogni $\alpha_i$ $\in$ P ed ogni \textit{REL}$_1$ $\in$ POS.\par
\vspace{0.3cm}
\noindent Ogni relazione spaziale \textit{REL}$_1$ dà informazioni sulla posizione relativa ad $\alpha_{i+1}$ rispetto ad $\alpha_{i+1}$. Nelle grammatiche tradizionali abbiamo una sola relazione spaziale che è data dalla concatenazione della stringa, nelle grammatiche posizionali possiamo definire altre relazioni spaziali ed in questo modo possiamo usarli per descrivere i linguaggi bidimensionali. In questi linguaggi, le informazioni posizionali vengono usate dal lexer per leggere il prossimo token.
\subsection{Relazioni spaziali}
L'insieme di relazioni spaziali \cite{pubblicazione: tomita} possono essere suddivisi in 3 sottoinsiemi: \textit{type}-2, \textit{type}-1, \textit{type}-0. Ogni sottoinsieme include quattro relazioni spaziali direzionali che sono: sopra, sotto, destra e sinistra. Ogni direzione può essere descritta da una funzione che prende in input una coordinata nel piano cartesiano e restituisce in output un insieme di posizioni. Infatti, se diamo in input una posizione \textit{p} nel piano, la relazione spaziale $\to$ in type-2 applicata a \textit{p} restituisce le prime posizioni a destra di \textit{p}; la relazione spaziale $\to$ in type-1 restituisce un insieme di posizioni a destra di \textit{p} e nella stessa linea di \textit{p}; la relazione spaziale $\to$ in type-0 restituisce un insieme di posizioni a destra di \textit{p} anche se non sono nella stessa posizione di \textit{p}. Qui di seguito mostriamo le definizioni dei tre tipi di relazioni spaziali ed illustreremo solo le funzioni descritte; le altre possono essere ricavate in maniera molto simile
\begin{flushleft}
	\textbf{Type-2}  \\
	$\to$   spostamento  di un movimento a destra: (\textit{x}, \textit{y}) $\to$ (\textit{x}+1, \textit{y})\\
	$\gets$   spostamento  di un movimento a sinistra \\
	$\uparrow$   spostamento  di un movimento in alto \\
	$\downarrow$   spostamento  di un movimento in basso \par
	\vspace{0.2cm}\noindent \textbf{Type-1}  \\
	$\to_1$   spostamento a destra sulla stessa linea:\par 
	\hspace{0.5cm}(\textit{x}, \textit{y}) $\to$ $\{$$\to^{(n)}$  (\textit{x},\textit{y}) $\mid$ n=1,2,$\dots$$\}$ \\
	$\gets_1$ spostamento a sinistra sulla stessa linea\\
	$\uparrow_1$ spostamento in alto sulla stessa linea\\
	$\downarrow_1$ spostamento in basso sulla stessa linea \par
	\vspace{0.2cm}\noindent \textbf{Type-0}  \\
	$\to_0$  spostamento a destra: \par 
	(\textit{x},\textit{y}) $\to$   $\to_0$(\textit{x},\textit{y})  $\cup$ $\downarrow_1$(\textit{x}$^{'}$,\textit{y}$^{'}$) $\cup$ $\to_1$ (\textit{x}$^{'}$, \textit{y}$^{'}$) dove (\textit{x}$^{'}$, \textit{y}$^{'}$) cambia in$\to_0$(\textit{x},\textit{y})\\
	$\gets_0$ spostamento  a sinistra \\
	$\uparrow_0$ spostamento in alto \\
	$\downarrow_0$ spostamento in basso 
\end{flushleft}
Nei nostri esempi utilizzeremo le relazioni spaziali HOR e VER che corrispondono alle relazioni spaziali $\to_0$ e $\downarrow_0$ - $\to_0$, rispettivamente, dove $\downarrow_0$ - $\to_0$ è un insieme che risulta essere la differenza tra l'insieme di posizioni generate da $\downarrow_0$ e $\to_0$ quando sono applicate alla stessa posizione. 
\subsection{Regole di valutazioni}
Una regola di valutazione PE \cite{pubblicazione: tomita} è una funzione che prende in input una stringa del tipo 
\begin{align}
	\textit{p}_1 REL_1 \textit{p}_2 REL_2 \dots REL_{m-1} \quad    m \ge 1 \notag
\end{align}
dove ogni \textit{p}$_i$ è una posizione ogni \textit{REL}$_i$ è una relazione spaziale, il suo output è una \textbf{immagine} dove gli elementi \textit{p}$_1$, \textit{p}$_2$,$\dots$, \textit{p}$_n$, sono disposti nello spazio in questo modo:
\begin{align}
	\textit{p}_i+1 \in REL_i (\textit{p}_i)  \quad    1\le \textit{i} \le \textit{m}-1 \notag
\end{align}
Le regole di valutazione delle relazioni spaziali sono state pensate per essere una sequenza da sinistra a destra. Diciamo che una regola di valutazione è \textbf{semplice} se non presenta effetti collaterali. \\
Un esempio di applicazione di semplici regole di valutazione sono i seguenti:
\begin{align}
	PE(a \to b \to c \to d) \quad \quad = a b c d \notag \\ 
		PE(a \quad VER \quad b \quad HOR \quad c) \quad  = \quad a \notag \\ 
		                                                              b \quad  c \notag 
\end{align}
\textbf{Derivazioni}\\
Denotiano $\alpha$ $\overset{*}{\Rightarrow}$ $\beta$ (si legge  $\alpha$ deriva inb zero o più passi $\beta$) se esiste una stringa $\alpha_0$$\alpha_1$$\dots$$\alpha_m$ (m$\ge$0) tale che
\begin{align}
	\alpha \Rightarrow \alpha_0 \Rightarrow \alpha_1 \Rightarrow \dots \Rightarrow  \alpha_m \quad = \quad \beta  \notag
\end{align}
La sequenza $\alpha_0$$\alpha_1$$\dots$$\alpha_m$ è chiamata \textbf{derivazione} di $\beta$ da $\alpha$. Una \textbf{forma sentenziale posizionale} è una stringa $\alpha$ tale che S $\overset{*}{\Rightarrow}$ $\alpha$. Una \textbf{frase posizionale} è una forma sentenziale posizionale con soli simboli terminali. Una \textbf{forma pittorica} è la valutazione di una forma sentenziale posizionale. Un \textbf{immagine} è una forma pittorica che non contiene non-terminali. Il \textbf{linguaggio pittorico L(G)} definito da una grammatica posizionale G è l'insieme delle sue immagini. Un esempio di grammatica posizionale è mostrato di seguito.
\begin{center}
	\textbf{N} = \{E,T,F\} \\
	\textbf{S} = E  \\
	\textbf{T} =  \{\textit{+}, \textit{hbar}, \textit{(}, \textit{)}, \textit{id}\} \\
	\textbf{POS}  = \{HOR, VER\}\\
	\textbf{PE} non è una semplice regola di valutazione\\
	\textbf{P}   = \{ E $\to$ E \textit{HOR} \textit{+}\textit{HOR} T $\mid$ T \par 
	         \hspace{1.1cm}T $\to$ \textit{VER} \textit{hbar} \textit{VER} F $\mid$ F \par 
	         \hspace{1.1cm} F $\to$ (\textit{HOR} E \textit{HOR}) $\mid$ \textit{id}  \}   
\end{center}
Una frase posizionale di questa grammatica è:
\begin{center}
	\textit{id} \textit{HOR} \textit{+} \textit{HOR} ( \textit{HOR} \textit{id} \textit{HOR} + \textit{HOR} \textit{id} \textit{HOR}) \textit{VER} \textit{hbar} \textit{VER} \textit{id} \textit{HOR} + \textit{HOR} + \textit{id}
\end{center}
Da questa immagine possiamo ottenere una delle possibili immagini:
\begin{center}
	 \textit{id} + $\frac{(\textit{id} + \textit{id})}{\textit{id}}$ +\textit{id}
\end{center}
\section{Parser Posizionali}
In questo paragrafo descriveremo il funzionamento dei parser posizionali e il modo che usano per gestire l'input.
\subsection{Gestione dell'input}
L'input che viene dato in pasto al parser è un immagine simbolica e ogni simbolo contenuto in esso è un token. L'input viene rappresentato da un array \textit{X} dove vengono memorizzati i token e da una matrice M che rappresenta l'input in base alla loro posizione spaziale e sono memorizzati con le posizioni \textit{i-esime} dell'array \textit{X}. Per segnalare la fine della stringa si aggiunge il simbolo $\$$ sia alla fine dell'array \textit{X} e sia nella matrice M.
\subsection{Gestione degli operatori spaziali}
Per ogni relazione spaziale definiamo un operatore spaziale con lo stesso nome. Ogni qualvolta il parser lo trova viene chiamata la funzione \textit{getNextToken()} che prende in input l'indice dell'array \textit{X} dell'ultimo token visto, l'operatore spaziale e un array \textit{Y} che contiene i token già visti. Restituisce il token successivo, dopo aver consultato la matrice \textit{M}. La ricerca sulla matrice avviene in questo modo:
\begin{itemize}
	\item Se l'operatore spaziale è \textit{HOR} (>) il calcolo del token successivo avviene in questo modo: si cerca un token non visto nelle colonne a destra della matrice M dell'ultimo token visto.
	\item Se invece l'operatore spaziale è \textit{VER} (<) il calcolo del token successivo avviene cercando un token non ancora visto nella righe successive dell'ultimo token visto all'interno della matrice \textit{M}.
\end{itemize}
La ricerca avviene usando l'array dei token visti e si restituisce il primo token che non è stato ancora visto.